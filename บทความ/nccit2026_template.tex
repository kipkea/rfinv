\documentclass[10pt,a4paper,twocolumn]{article}

% --- การตั้งค่าขอบกระดาษ 1 นิ้วทุกด้าน [cite: 17, 18, 27] ---
\usepackage[margin=1in, a4paper]{geometry}

% --- ฟอนต์และการตัดคำภาษาไทย [cite: 20, 29, 31] ---
\usepackage{fontspec}
\usepackage{xunicode}
\usepackage{xltxtra}

\usepackage[mode=buildnew]{standalone}
\usepackage{fontawesome5}

% ตั้งค่าฟอนต์ตาม NCCIT 2026
%\setmainfont{Times New Roman}
%\newfontfamily\thaifont{Angsana New}[Scale=1.25] % NCCIT กำหนด Angsana New
% *** เปลี่ยนชื่อฟอนต์ตรงนี้ถ้าเครื่องคุณใช้ฟอนต์อื่น ***
% เช่น Angsana New, Leelawadee UI, หรือ Browallia New
\setmainfont{Angsana New}[
    Scale=1.25,       % ปรับขนาดฟอนต์ให้อ่านง่ายขึ้นใน format IEEE
    BoldFont={* Bold},
    ItalicFont={* Italic},
    BoldItalicFont={* Bold Italic}
]
\newfontfamily\thaifont{Angsana New} % บรรทัดนี้สำคัญมาก เพื่อให้ TikZ เรียกใช้ได้

\usepackage[thai,english]{babel}

% --- การจัดการรูปภาพและโฟลเดอร์ [cite: 56] ---
\usepackage{graphicx}
\graphicspath{{pictures/}} % กำหนดโฟลเดอร์รูปภาพ

% --- การใช้งาน TikZ จากโฟลเดอร์ย่อย ---
\usepackage{standalone}
\usepackage{tikz}
\usetikzlibrary{positioning, shapes, fit}

% --- การอ้างอิง biblatex (สไตล์ IEEE ตามตัวอย่าง) [cite: 61, 64] ---
\usepackage[style=ieee, backend=biber]{biblatex}
\addbibresource{rfinv.bib}

% --- ตั้งค่าระยะบรรทัด [cite: 20, 31] ---
\usepackage{setspace}

% --- การจัดการหัวข้อ (Section) [cite: 38, 40] ---
\usepackage{titlesec}
\titleformat{\section}{\thaifont\fontsize{16}{19.2}\bfseries}{\thesection.}{0.5em}{}
\titleformat{\subsection}{\thaifont\fontsize{14}{16.8}\bfseries}{\thesubsection}{0.5em}{}

\XeTeXdashbreakstate=1
\usepackage{ragged2e} % ช่วยจัดการการจัดแนว

\begin{document}

% --- ชื่อบทความ (TH 20pt Bold, EN 14pt Bold) [cite: 1, 2, 23] ---
\twocolumn[
  \begin{center}
    {\thaifont\fontsize{20}{24}\bfseries ชื่อบทความภาษาไทย} \\
    \vspace{0.5em}
    {\fontsize{14}{16.8}\bfseries Research Title in English} \\
    \vspace{2em} % เว้นบรรทัดหลังชื่อบทความ [cite: 23]
    
    % --- ชื่อผู้แต่ง (TH 14pt Italic) [cite: 3, 24, 25] ---
    {\thaifont\fontsize{14}{16.8}\itshape ชื่อผู้แต่งไทย1 (English Name1)$^1$ และชื่อผู้แต่งไทย2 (English Name2)$^2$} \\
    {\thaifont\fontsize{14}{16.8}\itshape $^1$ชื่อภาควิชา, คณะ, มหาวิทยาลัย, $^2$ชื่อภาควิชา, คณะ, มหาวิทยาลัย} \\
    {\fontsize{10}{12}\itshape name1@anywhere.com, name2@anywhere.com} \\
    \vspace{1.5em}
  \end{center}
]

% --- บทคัดย่อภาษาไทย [cite: 7, 20, 21] ---
\begin{center}
    {\thaifont\fontsize{16}{19.2}\bfseries บทคัดย่อ}
\end{center}
{\thaifont\fontsize{14}{16.8}\itshape 
บทความนี้สรุปย่อเกี่ยวกับงานวิจัยที่ทำ โดยกล่าวถึงปัญหา วัตถุประสงค์ และผลการดำเนินงาน [cite: 21]
}
\par\vspace{0.5em}
{\thaifont\fontsize{14}{16.8}\bfseries{คำสำคัญ:} คำค้น1, คำค้น2}

% --- Abstract English [cite: 10, 20] ---
\begin{center}
    {\fontsize{12}{14.4}\bfseries Abstract}
\end{center}
\begin{spacing}{1.5} % ระยะบรรทัด 1.5 สำหรับ Abstract อังกฤษ [cite: 20]
{\fontsize{10}{12}\itshape 
This paper presents a guideline for preparing a paper for NCCIT. [cite: 11]
}
\end{spacing}
\par\vspace{0.5em}
{\thaifont\fontsize{10}{12}\bfseries Keywords: word 1, word 2}

\vspace{1em} % เว้นระยะก่อนเริ่มเนื้อหาหลัก [cite: 20]

% --- เนื้อหาหลัก [cite: 42] ---
\section{บทนำ}
\thaifont\fontsize{16}{19.2}บทความต้องไม่เกิน 6 หน้า  สามารถอ้างอิงรูปภาพจากโฟลเดอร์ย่อยได้ เช่น ภาพที่ \ref{fig:example} 

\begin{figure}[!ht]
    \centering
    \includegraphics[width=\linewidth]{box2.png} % ไฟล์อยู่ใน pictures/
    \caption{\thaifont\fontsize{12}{14.4} รายละเอียดภาพ }
    \label{fig:example}
\end{figure}

\section{วรรณกรรมที่เกี่ยวข้อง}
\thaifont\fontsize{16}{19.2}การอ้างอิงเอกสาร  เช่น งานวิจัยของ Meesad \cite{b_rfid_Automated_Billing} ซึ่งควรมีจำนวน 10-20 เรื่อง [cite: 46]

\section{วิธีการดำเนินการวิจัย}
\thaifont\fontsize{16}{19.2}การนำเข้าไฟล์ TikZ จากโฟลเดอร์ graphs:
\begin{figure}[!ht]
    \centering
    \includestandalone{graphs/rfinv_diagram_thai2} % ไม่ต้องใส่ .tex
    \caption{\thaifont\fontsize{12}{14.4} แผนผังการทำงานของระบบจัดการคลังสินค้า RFID}
\end{figure}

\section{สรุป}
\thaifont\fontsize{16}{19.2}สรุปผลการวิจัยและแนวทางในอนาคต [cite: 52]

% --- เอกสารอ้างอิง [cite: 60, 65] ---
\printbibliography[title={เอกสารอ้างอิง}]

\end{document}