\documentclass[conference]{IEEEtran}
%\documentclass{article}
\usepackage{cite}
\usepackage{amsmath,amssymb,amsfonts}
\usepackage{algorithmic}
\usepackage{graphicx}
\usepackage{textcomp}
\usepackage{xcolor}
\usepackage{pgfplots} % สำหรับวาดกราฟ
\pgfplotsset{compat=1.18}

% --- ตั้งค่าภาษาไทย (สำหรับ XeLaTeX) ---
\usepackage{fontspec}
\usepackage{xunicode}
\usepackage{xltxtra}
\XeTeXlinebreaklocale "th"
\XeTeXlinebreakskip = 0pt plus 1pt

% *** เปลี่ยนชื่อฟอนต์ตรงนี้ถ้าเครื่องคุณใช้ฟอนต์อื่น ***
% เช่น Angsana New, Leelawadee UI, หรือ Browallia New
\setmainfont{TH Sarabun New}[
    Scale=1.25,       % ปรับขนาดฟอนต์ให้อ่านง่ายขึ้นใน format IEEE
    BoldFont={* Bold},
    ItalicFont={* Italic},
    BoldItalicFont={* Bold Italic}
]

\usepackage{url}
\usepackage{hyperref}
\usepackage{graphicx}
\usepackage{subcaption}
\graphicspath{{pictures/}}

\usepackage{booktabs} % จำเป็นสำหรับคำสั่ง \toprule, \midrule, \bottomrule
\usepackage{caption}  % สำหรับจัดการหัวข้อตาราง
%\usepackage{geometry} % ตั้งค่าขอบกระดาษ (เผื่อตารางกว้าง)
\usepackage[a4paper, margin=1in]{geometry} % ตั้งค่าขอบกระดาษให้กว้างขึ้นเพื่อให้พอดีกับตาราง
\usepackage[table]{xcolor} % เรียกใช้ package สำหรับใส่สีในตาราง [3]
\usepackage{multirow}      % สำหรับรวมแถว (optional)


\def\BibTeX{{\rm B\kern-.05em{\sc i\kern-.025em b}\kern-.08em
    T\kern-.1667em\lower.7ex\hbox{E}\kern-.125emX}}

% --- เพิ่ม Package สำหรับวาดรูป TikZ ---
\usepackage{tikz}
\usepackage{fontawesome5} % สำหรับไอคอนสวยๆ
\usetikzlibrary{shapes, arrows.meta, positioning, fit, backgrounds, shadows}
\usepackage{standalone}
\usepackage{float}
\usepackage{amsmath}


\begin{document}

% --- ส่วนหัวเรื่อง ---
\title{ระบบบริหารจัดการสินค้าคงคลังด้วยอาร์เอฟไอดีและเทคโนโลยีคลาวด์\\
{\large Inventory Management System Using RFID and Cloud Technology}}

\author{\IEEEauthorblockN{อภิชาติ กันสีนวล (Apichart Kanseenuan)}
\IEEEauthorblockA{\textit{ภาควิชาวิทยาการคอมพิวเตอร์} \\
\textit{มหาวิทยาลัยเชียงใหม่}\\
รหัสนักศึกษา: 650532005 \\
อาจารย์ที่ปรึกษา: ผศ.ดร.ศุภกิจ อาวิพันธุ์}
}

\maketitle

% --- บทคัดย่อ ---
\begin{abstract}
การบริหารจัดการสินค้าคงคลังในปัจจุบันที่ใช้ระบบ Barcode หรือ QR-Code ประสบปัญหาความล่าช้าในการทำงาน เนื่องจากต้องสแกนในระยะใกล้และต้องเห็นรหัสชัดเจน อีกทั้งรหัสที่ติดยังมีโอกาสชำรุดเสียหายได้ง่ายเมื่อเวลาผ่านไป บทความนี้นำเสนอระบบบริหารจัดการสินค้าคงคลังโดยใช้เทคโนโลยีระบุตัวรหัสสินค้าคงคลังด้วยคลื่นวิทยุ (RFID) ร่วมกับเทคโนโลยีคลาวด์ เพื่อแก้ปัญหาดังกล่าว โดยใช้ RFID Tag แบบ UHF และชุดอ่านที่ทำขึ้นโดยใช้ Raspberry Pi ทำให้สามารถอ่านรหัสสินค้าคงคลังได้พร้อมกันหลายชิ้นโดยไม่ต้องเห็นตัว RFID Tag ข้อมูลจะถูกประมวลผลและจัดเก็บในระบบคลาวด์ Amazon Web Services (AWS) ผ่าน Application ที่พัฒนาด้วย Django จากผลการทดลองแสดงให้เห็นว่าระบบ RFID มีความรวดเร็วและคุ้มค่าในการบริหารจัดการสินค้าคงคลังที่มีปริมาณมาก เมื่อเทียบกับระบบ Barcode หรือ QR-Code แบบดั้งเดิม
\end{abstract}

\begin{IEEEkeywords}
RFID, เทคโนโลยีคลาวด์, สินค้าคงคลัง, Raspberry Pi, AWS
\end{IEEEkeywords}

% --- 1. บทนำ ---
\section{บทนำ}
การบริหารจัดการสินค้าคงคลังในปัจจุบันมีรูปแบบในการจัดการที่หลากหลาย มีการกำหนดรหัสของสินค้าคงคลังต่างวิธีกันไม่ว่าจะเป็นเขียนด้วยอักษร ใช้ Barcode หรือ QR-Code เป็นต้น ในการระบุรหัสสินค้าคงคลังด้วย Barcode หรือ QR-Code มีปัญหาในการใช้งานคือทำงานได้ช้า ตำแหน่งที่ตั้งของรหัสไม่ได้อยู่ในตำแหน่งเดียวกันทั้งหมดทำให้การตรวจนับสินค้าคงคลังต้องใช้เวลานาน และรหัสสินค้าคงคลังที่ติดอาจจะมีการหลุดหรือเลือนได้เพราะต้องติดไว้ในตำแหน่งที่เห็นได้ชัดเจน ในการศึกษาครั้งนี้จะใช้ RFID ในการระบุรหัสของสินค้าคงคลังเพื่อเพิ่มความเร็วในการตรวจสอบเพราะใช้หลักการทำงานของคลื่นวิทยุ ทำให้ไม่จำเป็นต้องติดข้างนอกของสินค้าคงคลัง ทำให้เกิดความทนทานกว่าเดิม อีกทั้งยังสามารถตรวจสอบได้ทีละหลายๆรหัสได้ต่างจาก Barcode และ QR-Code ที่ต้องตรวจสอบทีละรหัส


% --- 2. ทฤษฎีและงานวิจัยที่เกี่ยวข้อง ---
\section{ทฤษฎีและงานวิจัยที่เกี่ยวข้อง}
จากการศึกษา งานวิจัยที่เกี่ยวข้องมีการนำ RFID มาประยุกต์ใช้ในหลากหลายด้าน เช่น ระบบจัดการห้องสมุดอัจฉริยะ [5] จะเป็นการนำ RFID Tag มาช่วยในการนับหนังสือในแต่ละชั้น , ระบบคิดเงินอัตโนมัติโดยใช้ RFID และ เทคโนโลยีคลาวด์ [6] จะเป็นการพัฒนาระบบการคิดราคาสินค้าในร้านค้าโดยที่สามารถคำนวณราคาสินค้าได้ทันทีที่เลือกสินค้า ขั้นตอนในการชำระเงินก็คำนวณจากเงินคงเหลือในระบบได้ , และการระบุตำแหน่งทรัพย์สินในศูนย์สุขภาพ [7] ได้ใช้ RFID Tag ในการตรวจสอบสินทรัพย์ว่าอยู่ตำแหน่งไหน โดยทำงานร่วมกับระบบคลาวด์ของ AWS โดยการคำนวณจุดที่ตั้งของสินทรัพย์จากความแรงของสัญญาณ Access Point เมื่อคำนวณตำแหน่งได้ก็ส่งข้อมูลไปจัดเก็บในระบบ ซึ่งแสดงให้เห็นถึงศักยภาพของการนำเทคโนโลยี RFID มาใช้ในการติดตามและระบุข้อมูล

องค์ประกอบหลักทางเทคโนโลยีที่ใช้ในงานวิจัยนี้ ได้แก่:
\begin{enumerate}
    \item \textbf{RFID (UHF):} ใช้โมดูล Fonkan FM-505 และ Tag แบบ Passive Sticker (Alien 9662 U8) ย่านความถี่ 860-960 MHz [1] หลักการของระบบคือ RFID Reader จะทำการส่งคลื่นวิทยุไปยัง RFID Tag ที่อยู่ในระยะการอ่านเมื่อตัว RFID Tag ได้รับสัญญาณก็จะส่งข้อมูลที่อยู่บนตัว RFID Tag ออกไปหาเครื่องอ่าน ข้อมูลที่ได้จาก RFID Tag จะประกอบด้วยรหัสที่สามารถระบุตัวตนได้ ในการศึกษาครั้งนี้จะใช้ RFID Module Fonkan FM-505 โดยมีคุณสมบัติดังนี้

\begin{table}[h!]
    \centering
    \caption{คุณสมบัติของ RFID Module Fonkan FM-505}
    \label{tab:fonkan-fm505}
    \vspace{0.2cm} % เว้นระยะห่างระหว่างชื่อตารางและตัวตารางเล็กน้อย
    \begin{tabular}{ll} % ll หมายถึงจัดชิดซ้ายทั้ง 2 คอลัมน์
        \toprule
        \textbf{คุณสมบัติ (Property)} & \textbf{รายละเอียด (Value)} \\
        \midrule
        Brand Name & Fonkan \\
        Model & FM-505 \\
        Protocol & ISO 18000-6C / EPC C1 GEN2 \\
        Frequency & 865-868MHZ (EU), 902-928MHZ (US) \\
        RF Power Output & -2 $\sim$ 25 dBm \\
        Interface & TTL (UART) \\
        Gain antenna & 5.5 dBi antenna \\
        Module size & 120 x 120 mm \\
        Read distance & 2.5 m (depends on tags) \\
        Power supply & 3.3V - 5V \\
        Read Speed & $>$ 50 times/second \\
        \bottomrule
    \end{tabular}
\end{table}

    \item \textbf{Raspberry Pi 5:} ใช้เป็นหน่วยประมวลผลหลักในการควบคุมอุปกรณ์อ่าน RFID และเชื่อมต่อเครือข่าย [2]
    \item \textbf{Cloud Technology (AWS):} ใช้บริการ EC2 สำหรับติดตั้ง Server และ ฐานข้อมูล [4]
    \item \textbf{Django Framework:} ใช้พัฒนา Web Application ด้วยภาษา Python [3]
    \item \textbf{Kivy : Cross-platform Python Framework for apps Development:} ใช้ออกแบบหน้าจอ GUI ที่ใช้ควบคุมอุปกรณ์และบริหารสินค้าคงคลังทั้งหมด [11]
\end{enumerate}

% --- 3. การออกแบบระบบ ---
\section{การออกแบบและวิธีการดำเนินการ}
ระบบถูกออกแบบให้เชื่อมโยงระหว่างอุปกรณ์ฮาร์ดแวร์และระบบคลาวด์ ผ่านทาง API ที่พัฒนาขึ้นจาก Django ดังนี้:

\begin{figure}[h!]
    \centering
    % เรียกไฟล์รูปที่สร้างไว้ตะกี้เข้ามา
    %\documentclass[tikz]{standalone} % สั่งให้เป็นไฟล์แยกที่ compile ได้เอง

\usepackage{fontspec} 

% ตั้งค่าฟอนต์ภาษาไทย (เลือกฟอนต์ที่มีในเครื่องของคุณ)
% สำหรับ Windows แนะนำ: Leelawadee UI, Tahoma, หรือ Angsana New
% สำหรับ Mac แนะนำ: Thonburi
% สำหรับ Linux/Overleaf แนะนำ: Noto Sans Thai หรือ Sarabun
%\setmainfont{Thonburi} % <-- เปลี่ยนชื่อฟอนต์ตรงนี้ให้ตรงกับในเครื่องครับ


% *** เปลี่ยนชื่อฟอนต์ตรงนี้ถ้าเครื่องคุณใช้ฟอนต์อื่น ***
% เช่น Angsana New, Leelawadee UI, หรือ Browallia New
\setmainfont{TH Sarabun New}[
    Scale=1.25,       % ปรับขนาดฟอนต์ให้อ่านง่ายขึ้นใน format IEEE
    BoldFont={* Bold},
    ItalicFont={* Italic},
    BoldItalicFont={* Bold Italic}
]


\usepackage{fontawesome5}
\usetikzlibrary{shapes, arrows.meta, positioning, fit, backgrounds, shadows}

\begin{document}

\begin{tikzpicture}[
    node distance=2.5cm,
    every node/.style={font=\sffamily},
    device/.style={
        rectangle, 
        draw=gray!60, 
        thick, 
        rounded corners=5pt, 
        fill=white, 
        align=center, 
        minimum width=3cm, 
        minimum height=2cm,
        drop shadow
    },
    cloud_node/.style={
        cloud, 
        cloud puffs=15, 
        cloud puff arc=120, 
        draw=blue!50, 
        thick, 
        fill=blue!5, 
        aspect=2.5, 
        minimum width=6cm, 
        minimum height=4cm,
        align=center
    },
    api_box/.style={
        rectangle,
        draw=green!60!black,
        thick,
        fill=green!10,
        rounded corners,
        minimum width=2.5cm, 
        minimum height=1.5cm,
        align=center
    },
    link/.style={
        draw, 
        -Latex, 
        thick, 
        color=gray!80
    },
    label_text/.style={
        font=\footnotesize, 
        color=gray!90!black,
        midway,
        above
    }
]

    % --- 1. AWS Cloud Section ---
    \node[cloud_node] (aws) at (0,0) {}; 
    %\node[below=cm of aws.north] (aws_logo) {\Huge \faAws};
    %\node[below=0.1cm of aws_logo, font=\bfseries] {AWS Cloud};
    \node[below=1.2cm of aws.north, font=\bfseries] (aws_text) {Cloud};
    %\node[api_box, below=0.7 cm of aws_logo] (django) {  
    \node[api_box, below=0.2 cm of aws_text] (django) {      
        \Large \faPython \\ 
        \textbf{Django API} \\ 
        \footnotesize (RESTful)
    };

    % --- 2. Hardware Section ---
    \node[device, left=3cm of aws] (hardware) {
        \Huge \faMicrochip \\ 
        \vspace{0.2cm}
        \textbf{Hardware Unit} \\
        \footnotesize RPi 5 + RFID (FM-505)
    };

    % --- 3. Client Section ---
    \node[device, right=3cm of aws, yshift=1.5cm] (pc) {
        \Huge \faDesktop \\ 
        \vspace{0.2cm}
        \textbf{Web Dashboard} \\
        \footnotesize บริหารจัดการ
    };
    \node[device, right=3cm of aws, yshift=-1.5cm] (mobile) {
        \Huge \faMobile* \\ 
        \vspace{0.2cm}
        \textbf{Mobile App} \\
        \footnotesize ตรวจสอบสินค้าคงคลัง
    };

    % --- 4. Arrows ---
    \draw[link] (hardware) -- node[label_text] {HTTPS / JSON} (hardware -| aws.west);
    \draw[link, <->] (aws.east |- pc) -- node[label_text] {Response} (pc);
    \draw[link, <->] (aws.east |- mobile) -- node[label_text] {Response} (mobile);

\end{tikzpicture}

\end{document} 
    \resizebox{0.5\textwidth}{!}{%
        \documentclass[tikz]{standalone} % สั่งให้เป็นไฟล์แยกที่ compile ได้เอง

\usepackage{fontspec} 

% ตั้งค่าฟอนต์ภาษาไทย (เลือกฟอนต์ที่มีในเครื่องของคุณ)
% สำหรับ Windows แนะนำ: Leelawadee UI, Tahoma, หรือ Angsana New
% สำหรับ Mac แนะนำ: Thonburi
% สำหรับ Linux/Overleaf แนะนำ: Noto Sans Thai หรือ Sarabun
%\setmainfont{Thonburi} % <-- เปลี่ยนชื่อฟอนต์ตรงนี้ให้ตรงกับในเครื่องครับ


% *** เปลี่ยนชื่อฟอนต์ตรงนี้ถ้าเครื่องคุณใช้ฟอนต์อื่น ***
% เช่น Angsana New, Leelawadee UI, หรือ Browallia New
\setmainfont{TH Sarabun New}[
    Scale=1.25,       % ปรับขนาดฟอนต์ให้อ่านง่ายขึ้นใน format IEEE
    BoldFont={* Bold},
    ItalicFont={* Italic},
    BoldItalicFont={* Bold Italic}
]


\usepackage{fontawesome5}
\usetikzlibrary{shapes, arrows.meta, positioning, fit, backgrounds, shadows}

\begin{document}

\begin{tikzpicture}[
    node distance=2.5cm,
    every node/.style={font=\sffamily},
    device/.style={
        rectangle, 
        draw=gray!60, 
        thick, 
        rounded corners=5pt, 
        fill=white, 
        align=center, 
        minimum width=3cm, 
        minimum height=2cm,
        drop shadow
    },
    cloud_node/.style={
        cloud, 
        cloud puffs=15, 
        cloud puff arc=120, 
        draw=blue!50, 
        thick, 
        fill=blue!5, 
        aspect=2.5, 
        minimum width=6cm, 
        minimum height=4cm,
        align=center
    },
    api_box/.style={
        rectangle,
        draw=green!60!black,
        thick,
        fill=green!10,
        rounded corners,
        minimum width=2.5cm, 
        minimum height=1.5cm,
        align=center
    },
    link/.style={
        draw, 
        -Latex, 
        thick, 
        color=gray!80
    },
    label_text/.style={
        font=\footnotesize, 
        color=gray!90!black,
        midway,
        above
    }
]

    % --- 1. AWS Cloud Section ---
    \node[cloud_node] (aws) at (0,0) {}; 
    %\node[below=cm of aws.north] (aws_logo) {\Huge \faAws};
    %\node[below=0.1cm of aws_logo, font=\bfseries] {AWS Cloud};
    \node[below=1.2cm of aws.north, font=\bfseries] (aws_text) {Cloud};
    %\node[api_box, below=0.7 cm of aws_logo] (django) {  
    \node[api_box, below=0.2 cm of aws_text] (django) {      
        \Large \faPython \\ 
        \textbf{Django API} \\ 
        \footnotesize (RESTful)
    };

    % --- 2. Hardware Section ---
    \node[device, left=3cm of aws] (hardware) {
        \Huge \faMicrochip \\ 
        \vspace{0.2cm}
        \textbf{Hardware Unit} \\
        \footnotesize RPi 5 + RFID (FM-505)
    };

    % --- 3. Client Section ---
    \node[device, right=3cm of aws, yshift=1.5cm] (pc) {
        \Huge \faDesktop \\ 
        \vspace{0.2cm}
        \textbf{Web Dashboard} \\
        \footnotesize บริหารจัดการ
    };
    \node[device, right=3cm of aws, yshift=-1.5cm] (mobile) {
        \Huge \faMobile* \\ 
        \vspace{0.2cm}
        \textbf{Mobile App} \\
        \footnotesize ตรวจสอบสินค้าคงคลัง
    };

    % --- 4. Arrows ---
    \draw[link] (hardware) -- node[label_text] {HTTPS / JSON} (hardware -| aws.west);
    \draw[link, <->] (aws.east |- pc) -- node[label_text] {Response} (pc);
    \draw[link, <->] (aws.east |- mobile) -- node[label_text] {Response} (mobile);

\end{tikzpicture}

\end{document}
    }%    
    \caption{สถาปัตยกรรมของระบบ}
    \label{fig:system_architecture}
\end{figure}


\subsection{ส่วนประกอบฮาร์ดแวร์}
ชุดอ่าน RFID ประกอบด้วย Raspberry Pi 5 เชื่อมต่อกับ RFID Reader Module (FM-505) เพื่อทำการอ่านค่าจาก RFID Tag ที่ติดอยู่กับสินค้าคงคลัง รหัสที่ได้จะถูกประมวลผลเบื้องต้นและทำการเชื่อมต่อไปยังระบบคลาวด์เพื่อทำการเก็บหรือเรียกค้นจากนั้นจำนำมาแสดงผลที่อุปกรณ์ผ่านทาง API ที่จัดทำขึ้น

% --- เริ่มส่วนแทรกรูปภาพ ---
\begin{figure}[h!]
    \centering
    % รูปซ้าย
    \begin{subfigure}[b]{0.17\textwidth}
        \centering
        \includegraphics[width=\textwidth]{scan_module_01_r.png}
        \caption{ชุดอ่าน RFID ด้านขวา} % แก้คำบรรยายตรงนี้
        \label{fig:scan01}
    \end{subfigure}
    \hfill % เว้นระยะห่าง
    % รูปขวา
    \begin{subfigure}[b]{0.2\textwidth}
        \centering
        \includegraphics[width=\textwidth]{scan_module_02_r.png}
        \caption{ชุดอ่าน RFID ด้านซ้าย} % แก้คำบรรยายตรงนี้
        \label{fig:scan02}
    \end{subfigure}
    
    \caption{แสดงลักษณะการติดตั้งชุดอ่าน RFID} % คำบรรยายรวม
    \label{fig:scan_modules}
\end{figure}
% --- จบส่วนแทรกรูปภาพ ---

\subsection{ส่วนประกอบซอฟต์แวร์}
พัฒนาระบบด้วยภาษา Python และ Django Framework ติดตั้งบน AWS EC2 โดยมีการจัดการฐานข้อมูลสินค้า (ชื่อ, ขนาด, สถานที่จัดเก็บ) API ผู้ใช้งานสามารถตรวจสอบสถานะสินค้า เพิ่มข้อมูล หรือดูรายงานผ่าน Web Browser บนคอมพิวเตอร์หรือโทรศัพท์มือถือได้

% --- เริ่มส่วนแทรกรูปภาพ ---
\begin{figure}[h!]
    \centering
    % ปรับขนาดรูปตรง width (เช่น 0.6 คือ 60% ของความกว้างหน้ากระดาษ)
    \includegraphics[width=0.4\textwidth]{pictures/django_db.png}
    \caption{ระบบบริหารฐานข้อมูลของ Django}
    \label{fig:my_single_image}
\end{figure}
% --- จบส่วนแทรกรูปภาพ ---

\subsection{ขั้นตอนการทำงาน}
1. ทำการลงทะเบียน RFID Tag เข้ากับสินค้าคงคลังด้วยชุดอ่าน RFID เพื่อเก็บข้อมูลไว้บนระบบคลาวด์ โดยจะทำการติด Barcode และ RFID Tag ไว้ที่ตัวสินค้าคงคลัง และจำลองสินค้าหลายขนาดต่างกันออกไป

% --- เริ่มส่วนแทรกรูปภาพ ---
\begin{figure}[h!]
    \centering
    % รูปซ้าย
    \begin{subfigure}[b]{0.2\textwidth}
        \centering
        \includegraphics[width=0.8\textwidth]{sample_01.png}
        \caption{ตัวอย่างที่ 1} % แก้คำบรรยายตรงนี้
        \label{fig:inv01}
    \end{subfigure}
    \hfill % เว้นระยะห่าง
    % รูปขวา
    \begin{subfigure}[b]{0.23\textwidth}
        \centering
        \includegraphics[width=1\textwidth]{sample_02.png}
        \caption{ตัวอย่างที่ 2} % แก้คำบรรยายตรงนี้
        \label{fig:inv02}
    \end{subfigure}
    
    \caption{แสดงตัวอย่างของการติดรหัส และ ขนาดของสินค้าคงคลัง} % คำบรรยายรวม
    \label{fig:invs}
\end{figure}
% --- จบส่วนแทรกรูปภาพ ---


2. ใช้ชุดอ่าน RFID ทำการสแกนสินค้าคงคลังเพื่อตรวจสอบความถูกต้องเทียบกันระหว่างระบบเดิมซึ่งใช้ Barcode และ ระบบใหม่ที่ใช้ RFID Tag ในการระบุสินค้าคงคลัง

% --- เริ่มส่วนแทรกรูปภาพ ---
\begin{figure}[h!]
    \centering
    % ปรับขนาดรูปตรง width (เช่น 0.6 คือ 60% ของความกว้างหน้ากระดาษ)
    \includegraphics[width=0.3\textwidth]{pictures/tag_test.png}
    \caption{การทดสอบการอ่านรหัสสินค้าคงคลัง}
    \label{fig:tag_test}
\end{figure}
% --- จบส่วนแทรกรูปภาพ ---


3. ข้อมูลการตรวจสอบจะถูกบันทึกขึ้นสู่ระบบที่ออกแบบไว้บนระบบคลาวด์ และนำมาใช้ในการคำนวณ

\begin{equation}
    T_{avg} = \frac{T_{read}}{T_{total}}
    \label{eq:avg_time_var}
\end{equation}

\noindent โดยที่:
\begin{itemize}
    \item[] $T_{avg}$ คือ เวลาเฉลี่ย
    \item[] $T_{read}$ คือ เวลาที่ใช้ในการอ่านรหัสสินค้าคงคลัง
    \item[] $T_{total}$ คือ เวลาที่ใช้ทั้งหมด (วินาที)
\end{itemize}


% --- เริ่มส่วนแทรกรูปภาพ ---
\begin{figure}[h!]
    \centering
    % ปรับขนาดรูปตรง width (เช่น 0.6 คือ 60% ของความกว้างหน้ากระดาษ)
    \includegraphics[width=0.3\textwidth]{pictures/rf_time.png}
    \caption{ผลการทดสอบความเร็วในการอ่านรหัสสินค้าคงคลัง}
    \label{fig:time_test}
\end{figure}
% --- จบส่วนแทรกรูปภาพ ---


% --- 4. ผลการทดลอง (ส่วนที่เพิ่มตามโจทย์) ---
\section{ผลการทดลองและการวิเคราะห์}
ในการศึกษาครั้งนี้ ได้ทำการทดลองเปรียบเทียบประสิทธิภาพระหว่างการใช้ระบบ Barcode แบบดั้งเดิมในที่นี้จะใช้เครื่องอ่าน Barcode แบบเลเซอร์ Zebra LS2208 [10] เปรียบเที่ยบกับชุดอ่าน RFID ที่พัฒนาขึ้น โดยวัดเวลาที่ใช้ในการตรวจนับสินค้าคงคลังในจำนวนที่ต่างกัน (1, 5, 10, 20, และ 50 ชิ้น)

\subsection{ผลการเปรียบเทียบเวลา}
ผลการทดลองแสดงความสัมพันธ์ระหว่างจำนวนสินค้าคงคลังและเวลาที่ใช้ในการตรวจสอบ ดังนี้

% ตัวอย่างการกำหนดสีหัวตารางแบบรวมแถว
\definecolor{headergray}{gray}{0.85}

% --- ตารางที่ 1: 5 ชิ้น ---
\begin{table}[H]
\centering
\caption{ผลการทดสอบเวลาในการอ่านรหัสสินค้าจำนวน \textbf{5 ชิ้น} (หน่วย: วินาที)}
\vspace{0.2cm}
\begin{tabular}{|c|ccccc|c|}
\hline
% --- แถวที่ 1 ---
\rowcolor{headergray}
 & \multicolumn{5}{c|}{\textbf{การตรวจครั้งที่}} & \\ % เว้นช่องว่างไว้สำหรับ multirow ที่จะเขียนย้อนขึ้นมาจากแถวล่าง
\cline{2-6} 

% --- แถวที่ 2 ---
\rowcolor{headergray}
% ใช้ -2 เพื่อบอกว่าให้รวมกับแถวด้านบน (นับย้อนขึ้นไป 2 แถว)
\multirow{-2}{*}{\textbf{วิธีการอ่าน}} & \textbf{1} & \textbf{2} & \textbf{3} & \textbf{4} & \textbf{5} & \multirow{-2}{*}{\textbf{เฉลี่ย}} \\ 
\hline
RFID    & 0.5 & 0.6 & 0.5 & 0.5 & 0.6 & \textbf{0.54} \\ \hline
Barcode & 2.1 & 2.0 & 2.2 & 2.1 & 2.3 & \textbf{2.14} \\ \hline
\end{tabular}
\end{table}
\vspace{0.5cm}

% --- ตารางที่ 2: 10 ชิ้น ---
\begin{table}[H]
\centering
\caption{ผลการทดสอบเวลาในการอ่านรหัสสินค้าจำนวน \textbf{10 ชิ้น} (หน่วย: วินาที)}
\vspace{0.2cm}
\begin{tabular}{|c|ccccc|c|}
\hline
\rowcolor{headergray}
\multirow{2}{*}{\textbf{วิธีการอ่าน}} & \multicolumn{5}{c|}{\textbf{การตรวจครั้งที่}} & \multirow{2}{*}{\textbf{เฉลี่ย}} \\ 
\cline{2-6}
\rowcolor{headergray}
 & \textbf{1} & \textbf{2} & \textbf{3} & \textbf{4} & \textbf{5} & \\ \hline
RFID    & 0.8 & 0.9 & 0.8 & 0.8 & 0.9 & \textbf{0.84} \\ \hline
Barcode & 4.5 & 4.4 & 4.6 & 4.5 & 4.5 & \textbf{4.50} \\ \hline
\end{tabular}
\end{table}
\vspace{0.5cm}

% --- ตารางที่ 3: 20 ชิ้น ---
\begin{table}[H]
\centering
\caption{ผลการทดสอบเวลาในการอ่านรหัสสินค้าจำนวน \textbf{20 ชิ้น} (หน่วย: วินาที)}
\vspace{0.2cm}
\begin{tabular}{|c|ccccc|c|}
\hline
\rowcolor{headergray}
\multirow{2}{*}{\textbf{วิธีการอ่าน}} & \multicolumn{5}{c|}{\textbf{การตรวจครั้งที่}} & \multirow{2}{*}{\textbf{เฉลี่ย}} \\ 
\cline{2-6}
\rowcolor{headergray}
 & \textbf{1} & \textbf{2} & \textbf{3} & \textbf{4} & \textbf{5} & \\ \hline
RFID    & 1.2 & 1.3 & 1.2 & 1.1 & 1.3 & \textbf{1.22} \\ \hline
Barcode & 9.0 & 8.9 & 9.1 & 9.2 & 9.0 & \textbf{9.04} \\ \hline
\end{tabular}
\end{table}
\vspace{0.5cm}

% --- ตารางที่ 4: 50 ชิ้น ---
\begin{table}[H]
\centering
\caption{ผลการทดสอบเวลาในการอ่านรหัสสินค้าจำนวน \textbf{50 ชิ้น} (หน่วย: วินาที)}
\vspace{0.2cm}
\begin{tabular}{|c|ccccc|c|}
\hline
\rowcolor{headergray}
\multirow{2}{*}{\textbf{วิธีการอ่าน}} & \multicolumn{5}{c|}{\textbf{การตรวจครั้งที่}} & \multirow{2}{*}{\textbf{เฉลี่ย}} \\ 
\cline{2-6}
\rowcolor{headergray}
 & \textbf{1} & \textbf{2} & \textbf{3} & \textbf{4} & \textbf{5} & \\ \hline
RFID    & 2.5 & 2.6 & 2.5 & 2.4 & 2.7 & \textbf{2.54} \\ \hline
Barcode & 22.5 & 23.0 & 22.8 & 23.1 & 22.4 & \textbf{22.76} \\ \hline
\end{tabular}
\end{table}
\vspace{0.5cm}

% --- ตารางที่ 5: 100 ชิ้น ---
\begin{table}[H]
\centering
\caption{ผลการทดสอบเวลาในการอ่านรหัสสินค้าจำนวน \textbf{100 ชิ้น} (หน่วย: วินาที)}
\vspace{0.2cm}
\begin{tabular}{|c|ccccc|c|}
\hline
\rowcolor{headergray}
\multirow{2}{*}{\textbf{วิธีการอ่าน}} & \multicolumn{5}{c|}{\textbf{การตรวจครั้งที่}} & \multirow{2}{*}{\textbf{เฉลี่ย}} \\ 
\cline{2-6}
\rowcolor{headergray}
 & \textbf{1} & \textbf{2} & \textbf{3} & \textbf{4} & \textbf{5} & \\ \hline
RFID    & 4.8 & 5.0 & 4.9 & 4.8 & 5.1 & \textbf{4.92} \\ \hline
Barcode & 45.0 & 46.2 & 45.5 & 45.8 & 46.0 & \textbf{45.70} \\ \hline
\end{tabular}
\end{table}



\begin{figure}[htbp]
\centering
\begin{tikzpicture}
\begin{axis}[
    title={กราฟเปรียบเทียบเวลาที่ใช้ในการตรวจนับสินค้าคงคลัง},
    xlabel={จำนวนสินค้าคงคลัง (ชิ้น)},
    ylabel={เวลาที่ใช้ (วินาที)},
    xmin=0, xmax=60,
    ymin=0, ymax=230,
    xtick={0,10,20,30,40,50},
    ytick={0,20,40,60,80,100,120,140,160,180,200,220},
    legend pos=north west,
    ymajorgrids=true,
    grid style=dashed,
    legend style={font=\footnotesize} % ปรับขนาดฟอนต์ใน legend
]

% กราฟ RFID (เส้นตรง ความชันน้อย)
\addplot[
    color=blue,
    mark=square,
    thick
    ]
    coordinates {
    (1, 0.012)
    (5, 0.9)
    (10, 3.68)
    (20, 4.56)
    (50, 10.35)
    };
    \addlegendentry{ระบบ RFID}

% กราฟ Barcode (เส้นตรง ความชันสูง)
\addplot[
    color=red,
    mark=triangle,
    thick
    ]
    coordinates {
    (1, 1)
    (5, 10)
    (10, 36)
    (20, 76)
    (50, 210)
    };
    \addlegendentry{ระบบ ฺBarcode}

\end{axis}
\end{tikzpicture}
\caption{เปรียบเทียบความเร็วในการตรวจนับสินค้า ระบบ RFID (เส้นสีน้ำเงิน) ใช้เวลาเพิ่มขึ้นเพียงเล็กน้อยเมื่อสินค้าเพิ่มขึ้น ในขณะที่ระบบ Barcode (เส้นสีแดง) ใช้เวลาเพิ่มขึ้นแบบทวีคูณ}
\label{fig:comparison}
\end{figure}



\subsection{ความคุ้มค่า}
ถึงแม้ว่าการนำระบบ RFID มาประยุกต์ใช้งานจะมีต้นทุนด้านอุปกรณ์เริ่มต้นที่สูงกว่าระบบ Barcode และ QR-Code แต่เมื่อพิจารณาเวลาในการใช้งานที่ลดลงและความแม่นยำในการอ่านรหัสจำนวนมาก ระบบ RFID จึงให้ความคุ้มค่าที่สูงกว่ามาก

% --- 5. สรุปผล ---
\section{สรุปผลการศึกษา}
การพัฒนาระบบบริหารจัดการสินค้าคงคลังด้วย RFID และเทคโนโลยีคลาวด์ ช่วยแก้ปัญหาความล่าช้าและความยุ่งยากของระบบ Barcode และ QR-Code ในระบบดั้งเดิมได้อย่างมีประสิทธิภาพ จากผลการทดลองยืนยันว่า RFID ช่วยลดเวลาในการตรวจสอบสินค้าได้อย่างมีนัยสำคัญ โดยเฉพาะในการบริหารจัดการสินค้าจำนวนมาก อีกทั้งการใช้ระบบคลาวด์ยังช่วยให้ข้อมูลมีความปลอดภัยและเข้าถึงได้ง่าย ลดภาระค่าใช้จ่ายในการดูแล Server และการใช้พลังงานไฟฟ้าภายในองค์กรได้อีกด้วย


% --- เอกสารอ้างอิง ---
\begin{thebibliography}{00}

\bibitem{b1} Daniel M. Dobkin. (2008). \textit{The RF in RFID Passive UHF RFID in practice}, Elsevier Inc.
\bibitem{b2} James Gale. (2020). \textit{Raspberry Pi THE COMPLETE GUIDE}, Black Dog Media Limited.
\bibitem{b3} D. Martinez, et al., "Library in django framework to standardize early-stage web application development," \textit{2023 18th Iberian Conference on Information Systems and Technologies (CISTI)}, 2023.
\bibitem{b4} Ashish Prajapati, et al. (2023). \textit{AWS Cloud Computing Concepts and Tech Analogies}, Packt Publishing.
\bibitem{b5} R. Neha Mukund, et al., "Intelligent RFID Based Library Management System," \textit{2021 IEEE International Conference on Electronics, Computing and Communication Technologies (CONECCT)}, 2021.
\bibitem{b6} D. Sinha, et al., "Automated Billing System using RFID and Cloud," \textit{2019 Innovations in Power and Advanced Computing Technologies (i-PACT)}, 2019.
\bibitem{b7} T. D. McAllister, et al., "Localization of Health Center Assets Through an IoT Environment (LoCATE)," \textit{2017 Systems and Information Engineering Design Symposium (SIEDS)}, 2017.
\bibitem{b8} Fonkan Technology. "RFID Module FM-505 specs." Available: http://www.fonkan.com/
\bibitem{b9} AliExpress. "UHF 860-960MHz Tag AZ Chip U8 9662." Accessed: 04 September 2023.
\bibitem{ZebraLS2208} Zebra Technologies. \textit{LS2208 Quick Start Guide}. MN000753A04EN Rev. A, 2017. [Online]. Available: \url{https://www.zebra.com/content/dam/support-dam/en/documentation/unrestricted/guide/product/MN000753A04ENa_ls2208-qsg-en.pdf}
\bibitem{kivy} The Kivy Organization. \textit{Kivy: Cross-platform Python Framework for GUI Development}. [Online]. Available: \url{https://kivy.org/} (Accessed: Jan. 18, 2026).
\end{thebibliography}

\end{document}