%!TEX TS-program = xelatex
\documentclass[conference, a4paper]{IEEEtran}

% -----------------------------------------------------------
% 1. การตั้งค่าภาษาไทยและฟอนต์
% -----------------------------------------------------------
\usepackage{fontspec}
\usepackage{polyglossia}
\setdefaultlanguage{thai}

% ตั้งค่าฟอนต์หลัก (ภาษาอังกฤษ/ตัวเลข)
\setmainfont{THSarabun.ttf}[
    BoldFont = THSarabun Bold.ttf,
    ItalicFont = THSarabun Italic.ttf,
    BoldItalicFont = THSarabun BoldItalic.ttf
]

% ตั้งค่าฟอนต์สำหรับภาษาไทย (Polyglossia ต้องการตัวนี้)
\newfontfamily\thaifont{THSarabun.ttf}[
    BoldFont = THSarabun Bold.ttf,
    ItalicFont = THSarabun Italic.ttf,
    BoldItalicFont = THSarabun BoldItalic.ttf
]

\XeTeXlinebreaklocale "th"
\XeTeXlinebreakskip = 0pt plus 1pt

% -----------------------------------------------------------
% 2. เรียกใช้ Package และตั้งค่ารูปภาพ
% -----------------------------------------------------------
\usepackage{cite}
\usepackage{amsmath}
\usepackage{booktabs}
\usepackage{float}

\usepackage{graphicx}
% บอกให้ไปหารูปในโฟลเดอร์ picture ด้วย
\graphicspath{ {./pictures/} } 

% -----------------------------------------------------------
% 3. เริ่มต้นเอกสาร
% -----------------------------------------------------------
\begin{document}
% --- ชื่อบทความ ---
\title{การศึกษาเปรียบเทียบประสิทธิภาพความเร็วในการอ่านข้อมูล\\ระหว่างระบบ RFID และ Barcode ในงานคลังสินค้า}

% --- รายชื่อผู้เขียน ---
\author{
    \IEEEauthorblockN{ชื่อ-นามสกุล นักศึกษา}
    \IEEEauthorblockA{\textit{สาขาวิชาวิศวกรรมคอมพิวเตอร์} \\
    \textit{คณะวิศวกรรมศาสตร์ มหาวิทยาลัย...}\\
    จังหวัด, ประเทศไทย \\
    อีเมล: student@example.com}
    \and
    \IEEEauthorblockN{ชื่อ-นามสกุล อาจารย์ที่ปรึกษา}
    \IEEEauthorblockA{\textit{สาขาวิชาวิศวกรรมคอมพิวเตอร์} \\
    \textit{คณะวิศวกรรมศาสตร์ มหาวิทยาลัย...}\\
    จังหวัด, ประเทศไทย \\
    อีเมล: advisor@example.com}
}

\maketitle

% --- บทคัดย่อ (Abstract) ---
\begin{abstract}
งานวิจัยนี้มีวัตถุประสงค์เพื่อศึกษาและเปรียบเทียบประสิทธิภาพการทำงานระหว่างเทคโนโลยีระบบระบุตัวตนด้วยคลื่นวิทยุ (RFID) ย่านความถี่ UHF และระบบรหัสแท่ง (Barcode) โดยมุ่งเน้นที่ตัวแปรด้านความเร็วในการอ่านข้อมูล (Reading Speed) และความแม่นยำ

ผลการทดลองพบว่าระบบ RFID สามารถอ่านข้อมูลสินค้าจำนวน 500 ชิ้นได้ในเวลาเฉลี่ย 15 วินาที ในขณะที่ระบบ Barcode ใช้เวลาเฉลี่ย 25 นาที เนื่องจากข้อจำกัดที่ต้องสแกนในระยะสายตา ผลการวิจัยสรุปได้ว่า เทคโนโลยี RFID มีประสิทธิภาพสูงกว่า Barcode อย่างมีนัยสำคัญทางสถิติ และเหมาะสมสำหรับการนำไปประยุกต์ใช้ในระบบบริหารจัดการสินค้าคงคลังสมัยใหม่
\end{abstract}

% --- คำสำคัญ (Keywords) ---
\begin{IEEEkeywords}
RFID, Barcode, Warehouse Management, Efficiency, Speed Test
\end{IEEEkeywords}

% -----------------------------------------------------------
% เนื้อหาหลัก
% -----------------------------------------------------------

\section{บทนำ (Introduction)}
ในปัจจุบันระบบการจัดการคลังสินค้ามีความสำคัญอย่างยิ่งต่อภาคอุตสาหกรรม การระบุตัวตนสินค้าที่รวดเร็วและแม่นยำเป็นปัจจัยหลักในการลดต้นทุน... เทคโนโลยีบาร์โค้ด (Barcode) มีข้อจำกัดเรื่องการอ่านแบบ Line-of-sight \cite{ref1} ในขณะที่ RFID เข้ามาแก้ปัญหานี้ด้วยการใช้คลื่นวิทยุ...

\section{วิธีการดำเนินงาน (Methodology)}
การทดลองนี้แบ่งออกเป็น 2 ส่วนหลัก คือการทดสอบความเร็วและการทดสอบความแม่นยำ โดยใช้อุปกรณ์ดังนี้:
\begin{itemize}
    \item เครื่องอ่าน RFID Reader รุ่น UHF Gen 2 (920-925 MHz)
    \item เครื่องอ่าน Barcode Scanner แบบ Laser 1D
    \item สินค้าตัวอย่างจำนวน 500 ชิ้น
\end{itemize}

สมการคำนวณความเร็วเฉลี่ย ($V_{avg}$) คือ:
\begin{equation}
    V_{avg} = \frac{N_{total}}{T_{total}}
\end{equation}

\section{ผลการทดลอง (Experimental Results)}
จากการทดสอบเปรียบเทียบความเร็วระหว่างการใช้ Barcode สแกนทีละชิ้น กับการใช้ RFID เดินผ่านประตู (Gate Reader) ได้ผลลัพธ์ดังแสดงในรูปที่ \ref{fig:graph_comparison} และตารางที่ \ref{tab:comparison}

% --- ส่วนแทรกรูปภาพกราฟ ---
\begin{figure}[htbp]
    \centering
    % ดึงไฟล์รูปภาพชื่อ comparison_graph.png มาแสดง
    % width=\linewidth คือให้ความกว้างเท่ากับความกว้างคอลัมน์พอดี
    \includegraphics[width=\linewidth]{comparison_graph.png}    
    \caption{กราฟแสดงการเปรียบเทียบเวลาในการอ่านข้อมูลเมื่อจำนวนสินค้าเพิ่มขึ้น (Line Chart)}
    \label{fig:graph_comparison}
\end{figure}

จากกราฟจะเห็นได้ว่า เมื่อปริมาณสินค้าเพิ่มขึ้น เวลาที่ใช้ในระบบ Barcode จะเพิ่มขึ้นแบบเชิงเส้น (Linear) ตามจำนวนชิ้นสินค้า ในขณะที่ระบบ RFID ใช้เวลาเพิ่มขึ้นเพียงเล็กน้อยเนื่องจากสามารถอ่านได้พร้อมกันหลายชิ้น (Multi-read)

% --- ส่วนแทรกตารางเปรียบเทียบ ---
\begin{table}[htbp]
    \caption{ตารางเปรียบเทียบคุณสมบัติและประสิทธิภาพ}
    \begin{center}
    \begin{tabular}{lcc}
    \toprule % เส้นขอบบน (หนา)
    \textbf{รายการทดสอบ} & \textbf{RFID (Gen 2)} & \textbf{Barcode (1D)} \\
    \midrule % เส้นแบ่งหัวตาราง (บาง)
    ความเร็วเฉลี่ย (ชิ้น/วินาที) & 33.3 & 0.3 \\
    ระยะการอ่านสูงสุด (เมตร) & 5 - 8 & 0.2 - 0.5 \\
    การอ่านพร้อมกัน & ได้ (Anti-collision) & ไม่ได้ \\
    ความทนทานต่อสิ่งสกปรก & สูง & ต่ำ \\
    ต้นทุนต่อหน่วย (บาท) & 3 - 5 & < 0.1 \\
    \bottomrule % เส้นขอบล่าง (หนา)
    \end{tabular}
    \label{tab:comparison}
    \end{center}
\end{table}

\section{สรุปผล (Conclusion)}
จากการศึกษาพบว่า RFID มีความเร็วในการอ่านข้อมูลสูงกว่า Barcode อย่างมาก เหมาะสำหรับงานที่ต้องการความรวดเร็วและลดแรงงานคน...

% -----------------------------------------------------------
% เอกสารอ้างอิง
% -----------------------------------------------------------
\begin{thebibliography}{00}

\bibitem{ref1} 
G. Santucci, "The Internet of Things: Between the Revolution of the Internet and the Metamorphosis of Objects," in \textit{Vision and Challenges for Realising the Internet of Things}, European Commission, 2010.

\bibitem{ref2}
S. Shepard, \textit{RFID: Radio Frequency Identification}, McGraw-Hill Professional, 2005.

\end{thebibliography}

\end{document}