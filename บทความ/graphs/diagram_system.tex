% ไฟล์นี้เก็บเฉพาะโค้ดแผนภาพระบบ
\begin{figure*}[htbp] % ใช้ figure* เพื่อให้รูปขยายเต็มความกว้างสองคอลัมน์ (ถ้าเป็น format 2 คอลัมน์)
\centering
% เริ่มวาดรูป TikZ
\begin{tikzpicture}[node distance=2cm and 2.5cm] % ปรับระยะห่างระหว่างโหนด (แนวตั้ง and แนวนอน)

    % --- Zone 1: Warehouse Edge ---
    \node [hardware] (tags) {RFID Tags\\(ติดที่สินค้า)};
    \node [hardware, right=of tags] (reader) {UHF RFID\\Reader Module};

    % Raspberry Pi 5 (กล่องใหญ่ที่บรรจุ Logic ภายใน)
    \node [rpiblock, right=of reader, minimum width=3.5cm, minimum height=3cm] (rpi5) {};
    % หัวข้อ RPi5
    \node [above=0.1cm of rpi5.north, font=\bfseries] {Raspberry Pi 5 (Edge Gateway)};
    % Logic ภายใน RPi5
    \node [basicblock, fill=white, minimum width=3cm, below=0.5cm of rpi5.north, font=\scriptsize] (pyfilter) {Python Script\\(Data Filtering \& Debouncing)};
    \node [basicblock, fill=white, minimum width=3cm, above=0.5cm of rpi5.south, font=\scriptsize] (httpclient) {HTTP Client (requests)};

    % --- Zone Central: Internet ---
    \node [cloudshape, right=3cm of rpi5] (internet) {INTERNET};

    % --- Zone 2: AWS Cloud ---
    % AWS EC2 (กล่องใหญ่บรรจุ Django)
    \node [awsblock, right=3cm of internet, minimum width=3.5cm, minimum height=3cm] (ec2) {};
     % หัวข้อ EC2
    \node [above=0.1cm of ec2.north, font=\bfseries] {AWS EC2 (App Server)};
    % Logic ภายใน EC2
    \node [basicblock, fill=white, minimum width=3cm, minimum height=2cm] at (ec2.center) (django) {\textbf{Django REST Framework}\\(API Endpoint \& Logic)};

    % AWS RDS Database
    \node [database, below=1.5cm of ec2] (rds) {\textbf{Amazon RDS}\\(Database)};

    % --- Zone 3: Users ---
    \node [basicblock, below=2cm of internet] (users) {User Devices\\(Web/Mobile Dashboard)};


    % --- การเชื่อมต่อ (Connections) ---
    % Edge Zone connections
    \draw [dashedline] (tags) -- node[protoolabel] {คลื่นวิทยุ UHF} (reader);
    \draw [line, double] (reader) -- node[protoolabel] {USB / UART} (rpi5);
    % Internal RPi connections
    \draw [line, ->] (pyfilter) -- (httpclient);

    % API Connections to Cloud
    \draw [apiline] (rpi5.east) -- node[protoolabel] {HTTPS POST\\(JSON Data)} (internet.west);
    \draw [apiline] (internet.east) -- (ec2.west);

    % Database connection
    \draw [line, <->, thick] (django.south) -- node[protoolabel] {SQL Queries} (rds.north);

    % User connection
    \draw [dashedline, <->] (users.north) -- node[protoolabel] {HTTPS GET\\(เรียกดูข้อมูล)} (internet.south);


    % --- วาดกรอบแบ่งโซน (Background Layers) ---
    \begin{pgfonlayer}{background}
        % กรอบโซนคลังสินค้า
        \node [draw=gray!50, dashed, fill=gray!5, fit=(tags) (rpi5) (reader), label={[anchor=north west, font=\bfseries\small, xshift=0.5cm, yshift=-0.5cm]north west:Zone 1: Warehouse Edge (หน้างาน)}, inner sep=0.8cm, rounded corners] (zone1) {};

        % กรอบโซนคลาวด์
        \node [draw=orange!50, dashed, fill=orange!5, fit=(ec2) (rds), label={[anchor=north east, font=\bfseries\small, xshift=-0.5cm, yshift=-0.5cm]north east:Zone 2: Cloud}, inner sep=0.8cm, rounded corners] (zone2) {};
    \end{pgfonlayer}

\end{tikzpicture}
\caption{แผนภาพสถาปัตยกรรมระบบ แสดงการเชื่อมโยงจาก Raspberry Pi 5 ผ่านอินเทอร์เน็ตไปยัง Django API บน ระบบ Cloud}
\label{fig:system_architecture_tikz}
\end{figure*}