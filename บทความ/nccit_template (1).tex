\documentclass[a4paper,10pt,twocolumn]{article} % กำหนดเอกสารเป็น 2 คอลัมน์ [1]

% --- ส่วนนำ (Preamble) ---
\usepackage[utf8]{inputenc}
\usepackage{graphicx}       % สำหรับใส่รูปภาพ [2]
\usepackage{amsmath}        % สำหรับสมการคณิตศาสตร์ [3]
\usepackage{geometry}       % ปรับระยะขอบ (ปรับค่าตามข้อกำหนด NCCIT ได้)
\geometry{top=2.5cm, bottom=2.5cm, left=2cm, right=2cm}

% ข้อมูลบทความ
\title{\textbf{ชื่อบทความวิชาการ (Title of the Paper)}}
\author{ชื่อผู้แต่ง$^1$, ชื่อผู้แต่ง$^2$ \\
\small $^1$สาขาวิชา..., คณะ..., มหาวิทยาลัย... \\
\small $^2$สาขาวิชา..., คณะ..., มหาวิทยาลัย... \\
\small อีเมล: author1@example.com
}
\date{} % ไม่ใส่วันที่

\begin{document}

\maketitle % แสดงชื่อเรื่อง

% --- บทคัดย่อ ---
\begin{abstract}
บทคัดย่อควรสรุปใจความสำคัญของงานวิจัย วัตถุประสงค์ วิธีการ และผลลัพธ์อย่างย่อ โดยปกติจะมีความยาวประมาณ 150-250 คำ
\end{abstract}

\textbf{คำสำคัญ:} คำสำคัญ 1, คำสำคัญ 2, คำสำคัญ 3

% --- เนื้อหา ---
\section{บทนำ (Introduction)}
เนื้อหาส่วนบทนำ อธิบายความเป็นมาและความสำคัญของปัญหา ในการเขียนบทความวิชาการ คุณสามารถอ้างอิงเอกสารได้โดยใช้คำสั่ง cite เช่น การศึกษาเรื่อง AI มีความสำคัญมาก \cite{nccit_ref} หรืออ้างถึงหนังสือต้นฉบับ \cite{latex_guide} เป็นต้น [4]

\section{วิธีการดำเนินงาน (Methodology)}
อธิบายขั้นตอนการทำงาน หรือทฤษฎีที่ใช้ หากมีสมการคณิตศาสตร์ สามารถเขียนได้ดังนี้:
\begin{equation}
    E = mc^2
\end{equation}
และสามารถเขียนเมทริกซ์หรือสมการซับซ้อนได้โดยใช้ package amsmath [3]

\section{ผลการทดลอง (Results)}
แสดงผลข้อมูลในรูปแบบตาราง หรือกราฟ ตัวอย่างตาราง:

\begin{table}[h]
\centering
\begin{tabular}{|l|c|r|}
    \hline
    \textbf{การทดลอง} & \textbf{ค่า A} & \textbf{ค่า B} \\
    \hline
    ครั้งที่ 1 & 10.5 & 20.1 \\
    ครั้งที่ 2 & 11.0 & 19.8 \\
    \hline
\end{tabular}
\caption{ผลลัพธ์การทดลองเบื้องต้น}
\end{table}

\section{สรุปผล (Conclusion)}
สรุปสิ่งที่ได้จากการศึกษาและข้อเสนอแนะ

% --- บรรณานุกรม (References) ---
% เลือกรูปแบบการอ้างอิง เช่น plain, ieeetr (สำหรับสายวิศวะ/คอมฯ)
\bibliographystyle{ieeetr}  % [5]
\bibliography{rfinv}   % เรียกใช้ไฟล์ references.bib [5]

\end{document}