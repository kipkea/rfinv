\documentclass[conference]{IEEEtran}
\usepackage{cite}
\usepackage{amsmath,amssymb,amsfonts}
\usepackage{algorithmic}
\usepackage{graphicx}
\usepackage{textcomp}
\usepackage{xcolor}
\usepackage{pgfplots} % สำหรับวาดกราฟ
\pgfplotsset{compat=1.18}

% --- ตั้งค่าภาษาไทย (สำหรับ XeLaTeX) ---
\usepackage{fontspec}
\usepackage{xunicode}
\usepackage{xltxtra}
\XeTeXlinebreaklocale "th"
\XeTeXlinebreakskip = 0pt plus 1pt

% *** เปลี่ยนชื่อฟอนต์ตรงนี้ถ้าเครื่องคุณใช้ฟอนต์อื่น ***
% เช่น Angsana New, Leelawadee UI, หรือ Browallia New
\setmainfont{TH Sarabun New}[
    Scale=1.25,       % ปรับขนาดฟอนต์ให้อ่านง่ายขึ้นใน format IEEE
    BoldFont={* Bold},
    ItalicFont={* Italic},
    BoldItalicFont={* Bold Italic}
]

\def\BibTeX{{\rm B\kern-.05em{\sc i\kern-.025em b}\kern-.08em
    T\kern-.1667em\lower.7ex\hbox{E}\kern-.125emX}}

% --- เพิ่ม Package สำหรับวาดรูป TikZ ---
\usepackage{tikz}
\usetikzlibrary{shapes, arrows, positioning, fit, calc, backgrounds, shadows}

% --- กำหนดสไตล์ของกล่องและเส้นต่างๆ ---
\tikzset{
    % กล่องพื้นฐาน
    basicblock/.style = {rectangle, draw, rounded corners, minimum height=3em, minimum width=6em, align=center, font=\small, drop shadow},
    % กล่องฮาร์ดแวร์ (สีเทา)
    hardware/.style = {basicblock, fill=gray!10, draw=gray!60!black},
    % กล่อง Raspberry Pi (สีชมพู)
    rpiblock/.style = {basicblock, fill=magenta!10, draw=magenta!60!black},
    % กล่อง AWS/Cloud (สีส้ม)
    awsblock/.style = {basicblock, fill=orange!10, draw=orange!60!black},
    % กล่องฐานข้อมูล (รูปทรงกระบอก)
    database/.style = {cylinder, shape border rotate=90, draw, aspect=0.25, fill=blue!10, draw=blue!60!black, minimum height=4em, minimum width=4em, align=center, font=\small, drop shadow},
    % รูปก้อนเมฆอินเทอร์เน็ต
    cloudshape/.style = {cloud, draw, cloud puffs=10, cloud puff arc=120, aspect=2, fill=cyan!5, draw=cyan!60!black, minimum width=3.5cm, minimum height=2cm, align=center, font=\bfseries},
    % เส้นเชื่อมต่อ
    line/.style = {draw, -latex', thick, >={LaTeX[width=2mm,length=2mm]}},
    % เส้นประ
    dashedline/.style = {line, dashed},
    % เส้นหนาสำหรับ API
    apiline/.style = {line, line width=1.5pt, color=blue!70!black},
    % ป้ายกำกับบนเส้น
    protoolabel/.style = {midway, font=\footnotesize\sffamily, align=center, fill=white, inner sep=1.5pt, rounded corners}
}

\begin{document}

% --- ส่วนหัวเรื่อง ---
\title{ระบบบริหารจัดการสินค้าคงคลังด้วยอาร์เอฟไอดีและเทคโนโลยีคลาวด์\\
{\footnotesize Inventory Management System Using RFID and Cloud Technology}}

\author{\IEEEauthorblockN{อภิชาติ กันสีนวล (Apichart Kanseenuan)}
\IEEEauthorblockA{\textit{ภาควิชาวิทยาการคอมพิวเตอร์} \\
\textit{มหาวิทยาลัยเชียงใหม่}\\
รหัสนักศึกษา: 650532005 \\
อาจารย์ที่ปรึกษา: ผศ.ดร.ศุภกิจ อาวิพันธุ์}
}

\maketitle

% --- บทคัดย่อ ---
\begin{abstract}
การบริหารจัดการสินค้าคงคลังในปัจจุบันมีรูปแบบในการจัดการที่หลากหลาย มีการกำหนดรหัสของสินค้าคงคลังต่างวิธีกันไม่ว่าจะเป็นเขียนด้วยอักษร ใช้ Barcode หรือ QR-Code เป็นต้น ในการระบุรหัสสินค้าคงคลังด้วย Barcode หรือ QR-Code มีปัญหาในการใช้งานคือทำงานได้ช้า ตำแหน่งที่ตั้งของรหัสไม่ได้อยู่ในตำแหน่งเดียวกันทั้งหมดทำให้การตรวจนับสินค้าคงคลังต้องใช้เวลานาน และรหัสสินค้าคงคลังที่ติดอาจจะมีการหลุดหรือเลือนได้เพราะต้องติดไว้ในตำแหน่งที่เห็นได้ชัดเจน ในการศึกษาครั้งนี้จะใช้ RFID ในการระบุรหัสของสินค้าคงคลังเพื่อเพิ่มความเร็วในการตรวจสอบเพราะใช้หลักการทำงานของคลื่นวิทยุ ทำให้ไม่จำเป็นต้องติดข้างนอกของสินค้าคงคลัง ทำให้เกิดความทนทานกว่าเดิม อีกทั้งยังสามารถตรวจสอบได้ทีละหลายๆรหัสได้ต่างจาก Barcode และ QR-Code ที่ต้องตรวจสอบทีละรหัส
\end{abstract}

\begin{IEEEkeywords}
RFID, เทคโนโลยีคลาวด์, สินค้าคงคลัง, Raspberry Pi, AWS
\end{IEEEkeywords}

% --- 1. บทนำ ---
\section{บทนำ}
การบริหารจัดการสินค้าคงคลังเป็นหัวใจสำคัญของธุรกิจ ปัจจุบันมีการใช้รหัสสินค้าหลายรูปแบบ เช่น ตัวอักษร, Barcode หรือ QR-Code อย่างไรก็ตาม การใช้ Barcode มีข้อจำกัดสำคัญคือ ทำงานได้ช้าเนื่องจากตำแหน่งของรหัสอาจไม่อยู่ในจุดที่สะดวกต่อการตรวจสอบ และต้องใช้เวลาในการค้นหาตำแหน่งรหัสสินค้าแต่ละชิ้น นอกจากนี้ ฉลากรหัสยังมีโอกาสหลุดลอกหรือเลือนหายได้ง่ายเมื่อต้องติดในตำแหน่งที่สัมผัสได้

งานวิจัยนี้จึงนำเสนอการใช้เทคโนโลยี RFID (Radio Frequency Identification) เข้ามาช่วยในการระบุรหัสสินค้า เพื่อเพิ่มความเร็วในการตรวจสอบ เนื่องจากใช้คลื่นวิทยุในการรับส่งข้อมูล ทำให้ไม่จำเป็นต้องมองเห็นตัวรหัส (Non-line-of-sight) และสามารถอ่านข้อมูลได้ทีละหลายรหัสพร้อมกัน ซึ่งมีความทนทานและประสิทธิภาพสูงกว่า Barcode โดยระบบจะทำงานร่วมกับเทคโนโลยีคลาวด์ (Cloud Technology) เพื่อให้สามารถบริหารจัดการข้อมูลได้ทุกที่ผ่านอินเทอร์เน็ต

% --- 2. ทฤษฎีและงานวิจัยที่เกี่ยวข้อง ---
\section{ทฤษฎีและงานวิจัยที่เกี่ยวข้อง}
จากการศึกษา งานวิจัยที่เกี่ยวข้องมีการนำ RFID มาประยุกต์ใช้ในหลากหลายด้าน เช่น ระบบจัดการห้องสมุดอัจฉริยะ [5], ระบบคิดเงินอัตโนมัติ [6], และการระบุตำแหน่งทรัพย์สินในศูนย์สุขภาพ [7] ซึ่งแสดงให้เห็นถึงศักยภาพของ RFID ในการติดตามและระบุข้อมูล

องค์ประกอบหลักทางเทคโนโลยีที่ใช้ในงานวิจัยนี้ ได้แก่:
\begin{enumerate}
    \item \textbf{RFID (UHF):} ใช้โมดูล Fonkan FM-505 และ Tag แบบ Passive Sticker (Alien 9662 U8) ย่านความถี่ 860-960 MHz [1]
    \item \textbf{Raspberry Pi 3:} ใช้เป็นหน่วยประมวลผลหลักในการควบคุมอุปกรณ์อ่าน RFID และเชื่อมต่อเครือข่าย [2]
    \item \textbf{Cloud Technology (AWS):} ใช้บริการ EC2 สำหรับติดตั้ง Server และ RDS สำหรับฐานข้อมูล [4]
    \item \textbf{Django Framework:} ใช้พัฒนา Web Application ด้วยภาษา Python [3]
\end{enumerate}

% --- 3. การออกแบบระบบ ---
\section{การออกแบบและวิธีการดำเนินการ}
ระบบถูกออกแบบให้เชื่อมโยงระหว่างอุปกรณ์ฮาร์ดแวร์และระบบคลาวด์ ดังนี้:

\subsection{ส่วนประกอบฮาร์ดแวร์}
ชุดอ่าน RFID ประกอบด้วย Raspberry Pi 3 เชื่อมต่อกับ RFID Reader Module (FM-505) และเสาอากาศ เพื่อทำการอ่านค่าจาก RFID Tag ที่ติดอยู่กับสินค้า ข้อมูลที่ได้จะถูกประมวลผลเบื้องต้นและส่งต่อไปยังระบบคลาวด์

\subsection{ส่วนประกอบซอฟต์แวร์}
พัฒนาระบบด้วยภาษา Python และ Django Framework ติดตั้งบน AWS EC2 โดยมีการจัดการฐานข้อมูลสินค้า (ชื่อ, ขนาด, สถานที่จัดเก็บ) ผ่าน AWS RDS ผู้ใช้งานสามารถตรวจสอบสถานะสินค้า เพิ่มข้อมูล หรือดูรายงานผ่าน Web Browser บนคอมพิวเตอร์หรือโทรศัพท์มือถือ



\subsection{ขั้นตอนการทำงาน}
1. ติด RFID Tag เข้ากับสินค้าและลงทะเบียนข้อมูลเข้าระบบ
2. ใช้ชุดอ่านทำการสแกนสินค้าเพื่อตรวจสอบความถูกต้อง หรือนับจำนวน
3. ข้อมูลการตรวจสอบจะถูกบันทึกและอัปเดตสถานะขึ้นสู่ระบบ Cloud แบบ Real-time

% --- 4. ผลการทดลอง (ส่วนที่เพิ่มตามโจทย์) ---
\section{ผลการทดลองและการวิเคราะห์}
ในการศึกษาครั้งนี้ ได้ทำการทดลองเปรียบเทียบประสิทธิภาพระหว่างการใช้ระบบ Barcode แบบดั้งเดิม กับระบบ RFID ที่พัฒนาขึ้น โดยวัดเวลาที่ใช้ในการตรวจนับสินค้าคงคลังที่จำนวนต่างกัน (10, 50, 100, 200, และ 500 ชิ้น)

\subsection{ผลการเปรียบเทียบเวลา}
ผลการทดลองแสดงความสัมพันธ์ระหว่างจำนวนสินค้าและเวลาที่ใช้ในการตรวจสอบ ดังภาพที่ 1

\begin{figure}[htbp]
\centering
\begin{tikzpicture}
\begin{axis}[
    title={ภาพที่ 1 กราฟเปรียบเทียบเวลาในการตรวจนับสินค้า},
    xlabel={จำนวนสินค้า (ชิ้น)},
    ylabel={เวลาที่ใช้ (วินาที)},
    xmin=0, xmax=500,
    ymin=0, ymax=600,
    xtick={0,100,200,300,400,500},
    ytick={0,100,200,300,400,500,600},
    legend pos=north west,
    ymajorgrids=true,
    grid style=dashed,
    legend style={font=\footnotesize} % ปรับขนาดฟอนต์ใน legend
]

% กราฟ RFID (เส้นตรง ความชันน้อย)
\addplot[
    color=blue,
    mark=square,
    thick
    ]
    coordinates {
    (10, 2)
    (50, 5)
    (100, 8)
    (200, 12)
    (500, 20)
    };
    \addlegendentry{ระบบ RFID}

% กราฟ Barcode (Exponential/ความชันสูง)
\addplot[
    color=red,
    mark=triangle,
    thick,
    domain=10:500,
    samples=50
    ]
    {0.0025*x^2 + 0.5*x}; 
    % จำลองกราฟแบบ Exponential (Quadratic)
    \addlegendentry{ระบบ Barcode}

\end{axis}
\end{tikzpicture}
\caption{เปรียบเทียบความเร็วในการตรวจนับสินค้า ระบบ RFID (เส้นสีน้ำเงิน) ใช้เวลาเพิ่มขึ้นเพียงเล็กน้อยเมื่อสินค้าเพิ่มขึ้น ในขณะที่ระบบ Barcode (เส้นสีแดง) ใช้เวลาเพิ่มขึ้นแบบทวีคูณ}
\label{fig:comparison}
\end{figure}

จากกราฟจะเห็นได้ว่า:
\begin{enumerate}
    \item \textbf{ระบบ RFID:} เส้นกราฟมีความชันต่ำและเป็นลักษณะเส้นตรง (Linear) เนื่องจากสามารถอ่าน Tag ได้พร้อมกันหลายชิ้นในคราวเดียว ทำให้เวลาที่ใช้ในการตรวจสอบเพิ่มขึ้นเพียงเล็กน้อยแม้จำนวนสินค้าจะเพิ่มขึ้นมาก
    \item \textbf{ระบบ Barcode:} เส้นกราฟมีลักษณะพุ่งขึ้นแบบทวีคูณ (Exponential) เนื่องจากต้องทำการสแกนทีละชิ้น และต้องใช้เวลาในการจัดท่าทางหรือค้นหาตำแหน่งบาร์โค้ด ยิ่งจำนวนสินค้ามาก ความล่าช้าสะสมยิ่งสูงขึ้นอย่างชัดเจน
\end{enumerate}

\subsection{test graph}
เราได้ทำการทดลองและแสดงผลลัพธ์ดังภาพที่ \ref{fig:results_graph} และแผนภาพระบบในภาพที่ \ref{fig:system_arch}

% ==================================================================
% ตัวอย่างที่ 1: การดึงไฟล์กราฟผลการทดลองเข้ามา
% ==================================================================
\begin{figure}[htbp]
    \centering
    % ใช้คำสั่ง \input แล้วตามด้วยชื่อไฟล์กราฟที่เราแยกไว้
    % สมมติว่าไฟล์ชื่อ graph_results.tex และวางอยู่โฟลเดอร์เดียวกัน
    % ไฟล์นี้เก็บเฉพาะโค้ดกราฟผลการทดลอง
\begin{tikzpicture}
\begin{axis}[
    xlabel={จำนวนสินค้า (ชิ้น)},
    ylabel={เวลาที่ใช้ (วินาที)},
    xmin=0, xmax=500,
    ymin=0, ymax=600,
    xtick={0,100,200,300,400,500},
    ytick={0,100,200,300,400,500,600},
    legend pos=north west,
    ymajorgrids=true,
    grid style=dashed,
    legend style={font=\footnotesize},
    width=0.85\linewidth % ปรับขนาดให้พอดีกับคอลัมน์
]

% กราฟ RFID
\addplot[
    color=blue,
    mark=square,
    thick
    ]
    coordinates {
    (10, 1.5) (50, 4) (100, 6) (200, 10) (500, 18)
    };
    \addlegendentry{ระบบ RFID (RPi 5)}

% กราฟ Barcode
\addplot[
    color=red,
    mark=triangle,
    thick,
    domain=10:500,
    samples=50
    ]
    {0.0025*x^2 + 0.5*x}; 
    \addlegendentry{ระบบ Barcode}

\end{axis}
\end{tikzpicture}

    \caption{กราฟแสดงผลการทดลอง (โค้ดถูกแยกไปไว้ในไฟล์ graph\_results.tex)}
    \label{fig:results_graph}
\end{figure}

% ==================================================================
% ตัวอย่างที่ 2: การดึงไฟล์แผนภาพระบบเข้ามา
% ==================================================================
\begin{figure*}[htbp]
    \centering
    % ดึงอีกไฟล์เข้ามา
    % ไฟล์นี้เก็บเฉพาะโค้ดแผนภาพระบบ
\begin{figure*}[htbp] % ใช้ figure* เพื่อให้รูปขยายเต็มความกว้างสองคอลัมน์ (ถ้าเป็น format 2 คอลัมน์)
\centering
% เริ่มวาดรูป TikZ
\begin{tikzpicture}[node distance=2cm and 2.5cm] % ปรับระยะห่างระหว่างโหนด (แนวตั้ง and แนวนอน)

    % --- Zone 1: Warehouse Edge ---
    \node [hardware] (tags) {RFID Tags\\(ติดที่สินค้า)};
    \node [hardware, right=of tags] (reader) {UHF RFID\\Reader Module};

    % Raspberry Pi 5 (กล่องใหญ่ที่บรรจุ Logic ภายใน)
    \node [rpiblock, right=of reader, minimum width=3.5cm, minimum height=3cm] (rpi5) {};
    % หัวข้อ RPi5
    \node [above=0.1cm of rpi5.north, font=\bfseries] {Raspberry Pi 5 (Edge Gateway)};
    % Logic ภายใน RPi5
    \node [basicblock, fill=white, minimum width=3cm, below=0.5cm of rpi5.north, font=\scriptsize] (pyfilter) {Python Script\\(Data Filtering \& Debouncing)};
    \node [basicblock, fill=white, minimum width=3cm, above=0.5cm of rpi5.south, font=\scriptsize] (httpclient) {HTTP Client (requests)};

    % --- Zone Central: Internet ---
    \node [cloudshape, right=3cm of rpi5] (internet) {INTERNET};

    % --- Zone 2: AWS Cloud ---
    % AWS EC2 (กล่องใหญ่บรรจุ Django)
    \node [awsblock, right=3cm of internet, minimum width=3.5cm, minimum height=3cm] (ec2) {};
     % หัวข้อ EC2
    \node [above=0.1cm of ec2.north, font=\bfseries] {AWS EC2 (App Server)};
    % Logic ภายใน EC2
    \node [basicblock, fill=white, minimum width=3cm, minimum height=2cm] at (ec2.center) (django) {\textbf{Django REST Framework}\\(API Endpoint \& Logic)};

    % AWS RDS Database
    \node [database, below=1.5cm of ec2] (rds) {\textbf{Amazon RDS}\\(Database)};

    % --- Zone 3: Users ---
    \node [basicblock, below=2cm of internet] (users) {User Devices\\(Web/Mobile Dashboard)};


    % --- การเชื่อมต่อ (Connections) ---
    % Edge Zone connections
    \draw [dashedline] (tags) -- node[protoolabel] {คลื่นวิทยุ UHF} (reader);
    \draw [line, double] (reader) -- node[protoolabel] {USB / UART} (rpi5);
    % Internal RPi connections
    \draw [line, ->] (pyfilter) -- (httpclient);

    % API Connections to Cloud
    \draw [apiline] (rpi5.east) -- node[protoolabel] {HTTPS POST\\(JSON Data)} (internet.west);
    \draw [apiline] (internet.east) -- (ec2.west);

    % Database connection
    \draw [line, <->, thick] (django.south) -- node[protoolabel] {SQL Queries} (rds.north);

    % User connection
    \draw [dashedline, <->] (users.north) -- node[protoolabel] {HTTPS GET\\(เรียกดูข้อมูล)} (internet.south);


    % --- วาดกรอบแบ่งโซน (Background Layers) ---
    \begin{pgfonlayer}{background}
        % กรอบโซนคลังสินค้า
        \node [draw=gray!50, dashed, fill=gray!5, fit=(tags) (rpi5) (reader), label={[anchor=north west, font=\bfseries\small, xshift=0.5cm, yshift=-0.5cm]north west:Zone 1: Warehouse Edge (หน้างาน)}, inner sep=0.8cm, rounded corners] (zone1) {};

        % กรอบโซนคลาวด์
        \node [draw=orange!50, dashed, fill=orange!5, fit=(ec2) (rds), label={[anchor=north east, font=\bfseries\small, xshift=-0.5cm, yshift=-0.5cm]north east:Zone 2: Cloud}, inner sep=0.8cm, rounded corners] (zone2) {};
    \end{pgfonlayer}

\end{tikzpicture}
\caption{แผนภาพสถาปัตยกรรมระบบ แสดงการเชื่อมโยงจาก Raspberry Pi 5 ผ่านอินเทอร์เน็ตไปยัง Django API บน ระบบ Cloud}
\label{fig:system_architecture_tikz}
\end{figure*}

    \caption{แผนภาพระบบ (โค้ดถูกแยกไปไว้ในไฟล์ diagram\_system.tex)}
    \label{fig:system_arch}
\end{figure*}


\subsection{ความคุ้มค่า}
แม้ระบบ RFID จะมีต้นทุนอุปกรณ์เริ่มต้นที่สูงกว่า แต่เมื่อพิจารณาในระยะยาวและการจัดการสินค้าจำนวนมาก ระบบ RFID ให้ความคุ้มค่าสูงกว่ามากในด้านเวลาและแรงงานที่ลดลง

% --- 5. สรุปผล ---
\section{สรุปผลการศึกษา}
การพัฒนาระบบบริหารจัดการสินค้าคงคลังด้วย RFID และเทคโนโลยีคลาวด์ ช่วยแก้ปัญหาความล่าช้าและความยุ่งยากของระบบ Barcode เดิมได้อย่างมีประสิทธิภาพ ผลการทดลองยืนยันว่า RFID ช่วยลดเวลาในการตรวจสอบสินค้าได้อย่างมีนัยสำคัญ โดยเฉพาะในการบริหารจัดการสินค้าจำนวนมาก อีกทั้งการใช้ระบบคลาวด์ยังช่วยให้ข้อมูลมีความปลอดภัยและเข้าถึงได้ง่าย ลดภาระค่าใช้จ่ายในการดูแล Server ภายในองค์กร


% --- เอกสารอ้างอิง ---
\begin{thebibliography}{00}

\bibitem{b1} Daniel M. Dobkin. (2008). \textit{The RF in RFID Passive UHF RFID in practice}, Elsevier Inc.
\bibitem{b2} James Gale. (2020). \textit{Raspberry Pi THE COMPLETE GUIDE}, Black Dog Media Limited.
\bibitem{b3} D. Martinez, et al., "Library in django framework to standardize early-stage web application development," \textit{2023 18th Iberian Conference on Information Systems and Technologies (CISTI)}, 2023.
\bibitem{b4} Ashish Prajapati, et al. (2023). \textit{AWS Cloud Computing Concepts and Tech Analogies}, Packt Publishing.
\bibitem{b5} R. Neha Mukund, et al., "Intelligent RFID Based Library Management System," \textit{2021 IEEE International Conference on Electronics, Computing and Communication Technologies (CONECCT)}, 2021.
\bibitem{b6} D. Sinha, et al., "Automated Billing System using RFID and Cloud," \textit{2019 Innovations in Power and Advanced Computing Technologies (i-PACT)}, 2019.
\bibitem{b7} T. D. McAllister, et al., "Localization of Health Center Assets Through an IoT Environment (LoCATE)," \textit{2017 Systems and Information Engineering Design Symposium (SIEDS)}, 2017.
\bibitem{b8} Fonkan Technology. "RFID Module FM-505 specs." Available: http://www.fonkan.com/
\bibitem{b9} AliExpress. "UHF 860-960MHz Tag AZ Chip U8 9662." Accessed: 04 September 2023.

\end{thebibliography}

\end{document}