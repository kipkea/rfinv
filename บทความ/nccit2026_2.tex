\documentclass[a4paper,12pt]{article}
\usepackage{fontspec}
\usepackage{multicol}
\usepackage{graphicx}
\usepackage{geometry}
\usepackage{setspace}
\usepackage{titlesec}

% การตั้งค่าขนาดกระดาษ
\geometry{top=2.54cm, bottom=2.54cm, left=2.54cm, right=2.54cm}

% การตั้งค่าฟอนต์
\setmainfont{Times New Roman}
\newfontface\thaifont{Angsana New}

% การตั้งค่าหัวข้อ
\titleformat{\section}{\thaifont\Large\bfseries}{\thesection}{1em}{}
\titleformat{\subsection}{\thaifont\large\bfseries}{\thesubsection}{1em}{}

\begin{document}

% ชื่อบทความ
\begin{center}
    {\thaifont\huge ชื่อบทความ} \\[1em]
    {\fontspec{Times New Roman}\Large Title} \\
\end{center}

% ชื่อผู้แต่ง
\begin{center}
    {\thaifont\large ผู้แต่งไทย1 (English1)} \\[0.5em]
    {\thaifont\large ผู้แต่งไทย2 (English2)} \\[0.5em]
    {\thaifont\large ชื่อภาควิชาหรือหน่วยงาน ชื่อคณะ ชื่อมหาวิทยาลัย} \\[0.5em]
    {\thaifont\large ชื่อภาควิชาหรือหน่วยงาน ชื่อคณะ ชื่อมหาวิทยาลัย} \\[0.5em]
    \texttt{name1@anywhere.com, name2@anywhere.com}
\end{center}

% บทคัดย่อ
\section*{\thaifont บทคัดย่อ}
{\thaifont\normalsize
บทความนี้เป็นตัวอย่างสำหรับการเตรียมการเขียนบทความที่จะส่งให้คณะกรรมการพิจารณาลงพิมพ์ในเอกสารประกอบการประชุม NCCIT...
}
\vspace{1em}

\textbf{คำสำคัญ:} คำค้น1, คำค้น2, คำค้น3, คำค้น4, คำค้น5

% Abstract
\section*{Abstract}
{\fontspec{Times New Roman}\normalsize
This paper presents a guideline for preparing a paper to submit to the NCCIT committee...
}
\vspace{1em}

\textbf{Keywords:} word 1, word 2, word 3, word 4, word 5.

% บทนำ
\section{บทนำ}
บทความที่จะส่งต้องใช้กระดาษขนาด A4...

% รูปแบบบทความ
\section{รูปแบบบทความ}
\subsection{ขอบเขตกระดาษ}
เนื้อหาในบทความต้องอยู่ภายในขอบเขต...

% เนื้อหาหลัก
\section{เนื้อหาหลัก}
เนื้อหาหลักควรประกอบด้วยหัวข้อดังต่อไปนี้...

% การส่งบทความ
\section{การส่งบทความ}
ส่งบทความที่ได้รับการจัดรูปแบบ...

% เอกสารอ้างอิง
\section{เอกสารอ้างอิง}
\begin{thebibliography}{99}
\bibitem{ref1} P. P. Lin and K. Jules, “An intelligent system for monitoring the microgravity environment...”
\end{thebibliography}

\end{document}