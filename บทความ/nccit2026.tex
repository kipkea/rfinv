%\documentclass[conference]{IEEEtran}
%\documentclass{article}
\documentclass[twocolumn, a4paper]{article}

%\usepackage{cite}
\usepackage[backend=biber, style=ieee]{biblatex}
\addbibresource{rfinv.bib} % ตรวจสอบว่าชื่อไฟล์ .bib ตรงกัน
\usepackage{amsmath,amssymb,amsfonts}
\usepackage{algorithmic}
\usepackage{graphicx}
\usepackage{textcomp}
\usepackage{xcolor}
\usepackage{pgfplots} % สำหรับวาดกราฟ
\pgfplotsset{compat=1.18}

% --- ตั้งค่าภาษาไทย (สำหรับ XeLaTeX) ---
\usepackage{fontspec}
\usepackage{xunicode}
\usepackage{xltxtra}
\XeTeXlinebreaklocale "th"
\XeTeXlinebreakskip = 0pt plus 1pt

% *** เปลี่ยนชื่อฟอนต์ตรงนี้ถ้าเครื่องคุณใช้ฟอนต์อื่น ***
% เช่น Angsana New, Leelawadee UI, หรือ Browallia New
\setmainfont{Angsana New}[
    Scale=1.25,       % ปรับขนาดฟอนต์ให้อ่านง่ายขึ้นใน format IEEE
    BoldFont={* Bold},
    ItalicFont={* Italic},
    BoldItalicFont={* Bold Italic}
]
\newfontfamily\thaifont{Angsana New} % บรรทัดนี้สำคัญมาก เพื่อให้ TikZ เรียกใช้ได้


\usepackage{url}
\usepackage{hyperref}
\usepackage{graphicx}
\usepackage{subcaption}
\graphicspath{{pictures/}}

\usepackage{booktabs} % จำเป็นสำหรับคำสั่ง \toprule, \midrule, \bottomrule
\usepackage{caption}  % สำหรับจัดการหัวข้อตาราง
%\usepackage{geometry} % ตั้งค่าขอบกระดาษ (เผื่อตารางกว้าง)
%\usepackage[a4paper, margin=1in]{geometry} % ตั้งค่าขอบกระดาษให้กว้างขึ้นเพื่อให้พอดีกับตาราง
\usepackage[table]{xcolor} % เรียกใช้ package สำหรับใส่สีในตาราง [3]
\usepackage{multirow}      % สำหรับรวมแถว (optional)

% --- ส่วนที่ต้องเพิ่ม/แก้ไขเพื่อให้ตรงกับ NCCIT ---
\usepackage[a4paper, left=1.5cm, right=1.5cm, top=2cm, bottom=2.5cm]{geometry} % ขอบมาตรฐานโดยประมาณ
\setlength{\columnsep}{0.8cm} % ปรับระยะห่างระหว่างคอลัมน์ตรงนี้


\def\BibTeX{{\rm B\kern-.05em{\sc i\kern-.025em b}\kern-.08em
    T\kern-.1667em\lower.7ex\hbox{E}\kern-.125emX}}

% --- เพิ่ม Package สำหรับวาดรูป TikZ ---
\usepackage{tikz}
\usepackage{fontawesome5} % สำหรับไอคอนสวยๆ
\usetikzlibrary{shapes, arrows.meta, positioning, fit, backgrounds, shadows}
\usepackage{standalone}
\usepackage{float}
\usepackage{amsmath}
\usepackage{tabularx} % เรียกใช้ package tabularx [2]
\usepackage{colortbl}

% ตัวอย่างการกำหนดสีหัวตารางแบบรวมแถว
\definecolor{headergray}{gray}{0.85}


% กำหนดคอลัมน์ตัวเลขให้เล็กลงและจัดกึ่งกลาง
\newcolumntype{C}{>{\centering\arraybackslash\fontsize{8}{10}\selectfont}X}

\newcommand{\RFIDHalfTable}[8]{
    \begin{table}[H]
    \centering
    \caption{\small ผลการทดสอบเวลาในการอ่านรหัสสินค้าจำนวน #1 ชิ้น (หน่วย: วินาที)}
    \vspace{0.2cm}
    \renewcommand{\arraystretch}{1.5}
    \setlength{\tabcolsep}{2pt} % ลดระยะห่างระหว่างคอลัมน์เพื่อให้ลงตัวในครึ่งหน้า
    \begin{tabularx}{\columnwidth}{|>{\centering\arraybackslash\bfseries\fontsize{8}{10}\selectfont}p{1.3cm}|CCCCC|c|}
    \hline
    \rowcolor{headergray}
    \textbf{วิธี} & \textbf{ครั้ง 1} & \textbf{ครั้ง 2} & \textbf{ครั้ง 3} & \textbf{ครั้ง 4} & \textbf{ครั้ง 5} & \textbf{เฉลี่ย} \\ \hline
    RFID    & #2 & #3 & #4 & #5 & #6 & \textbf{#7} \\ \hline
    Barcode & #8 \\ \hline
    \end{tabularx}
    \end{table}
}

\newcommand{\BoxTable}[8]{
    \begin{table}[H]
    \centering
    \caption{\small ผลการทดสอบเวลาในการอ่านรหัสสินค้าจำนวน #1 ชิ้น (หน่วย: วินาที)}
    \vspace{0.2cm}
    \renewcommand{\arraystretch}{1.5}
    \setlength{\tabcolsep}{2pt} % ลดระยะห่างระหว่างคอลัมน์เพื่อให้ลงตัวในครึ่งหน้า
    \begin{tabularx}{\columnwidth}{|>{\centering\arraybackslash\bfseries\fontsize{8}{10}\selectfont}p{1.3cm}|CCCCC|c|}
    \hline
    \rowcolor{headergray}
    \textbf{ประเภท} & \textbf{ครั้ง 1} & \textbf{ครั้ง 2} & \textbf{ครั้ง 3} & \textbf{ครั้ง 4} & \textbf{ครั้ง 5} & \textbf{เฉลี่ย} \\ \hline
    กล่องกระดาษ    & #2 & #3 & #4 & #5 & #6 & \textbf{#7} \\ \hline
    กล่องพลาสติก & #8 \\ \hline
    \end{tabularx}
    \end{table}
}



\begin{document}

% --- ส่วนหัวเรื่อง ---
\title{ระบบบริหารจัดการสินค้าคงคลังด้วยอาร์เอฟไอดีและเทคโนโลยีคลาวด์\\
{\large Inventory Management System Using RFID and Cloud Technology}}

\author{{อภิชาติ กันสีนวล (Apichart Kanseenuan)}
\textit{ภาควิชาวิทยาการคอมพิวเตอร์} \\
\textit{มหาวิทยาลัยเชียงใหม่}\\
รหัสนักศึกษา: 650532005 \\
อาจารย์ที่ปรึกษา: ผศ.ดร.ศุภกิจ อาวิพันธุ์
}

\maketitle

% --- บทคัดย่อ ---
\begin{abstract}
การบริหารจัดการสินค้าคงคลังในปัจจุบันที่ใช้ระบบบาร์โค้ด หรือ คิวอาร์โค้ด ประสบปัญหาความล่าช้าในการทำงาน เนื่องจากต้องสแกนในระยะใกล้และต้องเห็นรหัสชัดเจน อีกทั้งรหัสที่ติดยังมีโอกาสชำรุดเสียหายได้ง่ายเมื่อเวลาผ่านไป บทความนี้นำเสนอระบบบริหารจัดการสินค้าคงคลังโดยใช้เทคโนโลยีระบุตัวรหัสสินค้าคงคลังด้วยคลื่นวิทยุ ร่วมกับเทคโนโลยีคลาวด์ เพื่อแก้ปัญหาดังกล่าว โดยใช้ แท็กอาร์เอฟไอดี แบบ UHF และชุดอ่านที่ทำขึ้นโดยใช้ Raspberry Pi ทำให้สามารถอ่านรหัสสินค้าคงคลังได้พร้อมกันหลายชิ้นโดยไม่ต้องเห็นตัวแท็กอาร์เอฟไอดี ข้อมูลจะถูกประมวลผลและจัดเก็บในระบบคลาวด์ ผ่านแอพลิเคชั่น ที่พัฒนาด้วยเฟรมเวิร์ค Django จากผลการทดลองแสดงให้เห็นว่าระบบอาร์เอฟไอดี มีความรวดเร็วและคุ้มค่าในการบริหารจัดการสินค้าคงคลังที่มีปริมาณมาก เมื่อเทียบกับระบบบาร์โค้ด หรือ คิวอาร์โค้ด  แบบดั้งเดิม
\end{abstract}
\textbf{Keywords:} Cloud Technology, Inventory, Raspberry Pi, Inventory, RFID


% --- 1. บทนำ ---
\section{บทนำ}
การบริหารจัดการสินค้าคงคลังในปัจจุบันมีรูปแบบในการจัดการที่หลากหลาย มีการกำหนดรหัสของสินค้าคงคลังต่างวิธีกันไม่ว่าจะเป็นเขียนด้วยอักษร ใช้ Barcode หรือ QR-Code เป็นต้น ในการระบุรหัสสินค้าคงคลังด้วย Barcode หรือ QR-Code มีปัญหาในการใช้งานคือทำงานได้ช้า ตำแหน่งที่ตั้งของรหัสไม่ได้อยู่ในตำแหน่งเดียวกันทั้งหมดทำให้การตรวจนับสินค้าคงคลังต้องใช้เวลานาน และรหัสสินค้าคงคลังที่ติดอาจจะมีการหลุดหรือเลือนได้เพราะต้องติดไว้ในตำแหน่งที่เห็นได้ชัดเจน ในการศึกษาครั้งนี้จะใช้ RFID ในการระบุรหัสของสินค้าคงคลังเพื่อเพิ่มความเร็วในการตรวจสอบเพราะใช้หลักการทำงานของคลื่นวิทยุ ทำให้ไม่จำเป็นต้องติดข้างนอกของสินค้าคงคลัง ทำให้เกิดความทนทานกว่าเดิม อีกทั้งยังสามารถตรวจสอบได้ทีละหลายๆรหัสได้ต่างจาก Barcode และ QR-Code ที่ต้องตรวจสอบทีละรหัส


% --- 2. ทฤษฎีและงานวิจัยที่เกี่ยวข้อง ---
\section{ทฤษฎีและงานวิจัยที่เกี่ยวข้อง}
จากการศึกษา งานวิจัยที่เกี่ยวข้องมีการนำ RFID มาประยุกต์ใช้ในหลากหลายด้าน เช่น ระบบจัดการห้องสมุดอัจฉริยะ  \cite{b_rfid_Library} จะเป็นการนำ RFID Tag มาช่วยในการนับหนังสือในแต่ละชั้น , ระบบคิดเงินอัตโนมัติโดยใช้ RFID และ เทคโนโลยีคลาวด์  \cite{b_rfid_Automated_Billing} จะเป็นการพัฒนาระบบการคิดราคาสินค้าในร้านค้าโดยที่สามารถคำนวณราคาสินค้าได้ทันทีที่เลือกสินค้า ขั้นตอนในการชำระเงินก็คำนวณจากเงินคงเหลือในระบบได้ , และการระบุตำแหน่งทรัพย์สินในศูนย์สุขภาพ  \cite{b_rfid_LoCATE} ได้ใช้ RFID Tag ในการตรวจสอบสินทรัพย์ว่าอยู่ตำแหน่งไหน โดยทำงานร่วมกับระบบคลาวด์ของ AWS โดยการคำนวณจุดที่ตั้งของสินทรัพย์จากความแรงของสัญญาณ Access Point เมื่อคำนวณตำแหน่งได้ก็ส่งข้อมูลไปจัดเก็บในระบบ ซึ่งแสดงให้เห็นถึงศักยภาพของการนำเทคโนโลยี RFID มาใช้ในการติดตามและระบุข้อมูล

องค์ประกอบหลักทางเทคโนโลยีที่ใช้ในงานวิจัยนี้ ได้แก่:
\begin{enumerate}
    \item \textbf{RFID (UHF):} ใช้โมดูล Fonkan FM-505 และ Tag แบบ Passive Sticker (Alien 9662 U8) ย่านความถี่ 860-960 MHz \cite{b_rfid_book} หลักการทำงานของระบบคือ RFID Reader จะทำการส่งคลื่นวิทยุไปยัง RFID Tag ที่อยู่ในระยะการอ่านเมื่อตัว RFID Tag ได้รับสัญญาณก็จะส่งข้อมูลที่อยู่บนตัว RFID Tag ออกไปหาเครื่องอ่าน ข้อมูลที่ได้จาก RFID Tag จะประกอบด้วยรหัสที่สามารถระบุตัวตนได้ ในการศึกษาครั้งนี้จะใช้ RFID Module Fonkan FM-505  \cite{b_rfid_fm505} และ RFID Tag LH-9662PST-100 \cite{b_rfid_tag} โดยมีคุณสมบัติดังนี้

\begin{table}[h!]
    \centering
    \caption{คุณสมบัติของ RFID Module Fonkan FM-505  \cite{b_rfid_fm505}}
    \label{tab:fonkan-fm505}
    \vspace{0.2cm} % เว้นระยะห่างระหว่างชื่อตารางและตัวตารางเล็กน้อย
    \begin{tabular}{ll} % ll หมายถึงจัดชิดซ้ายทั้ง 2 คอลัมน์
        \toprule
        \textbf{คุณสมบัติ (Property)} & \textbf{รายละเอียด (Value)} \\
        \midrule
        Brand Name & Fonkan \\
        Model & FM-505 \\
        Protocol & ISO 18000-6C / EPC C1 GEN2 \\
        Frequency & 865-868MHZ (EU), 902-928MHZ (US) \\
        RF Power Output & -2 $\sim$ 25 dBm \\
        Interface & TTL (UART) \\
        Gain antenna & 5.5 dBi antenna \\
        Module size & 120 x 120 mm \\
        Read distance & 2.5 m (depends on tags) \\
        Power supply & 3.3V - 5V \\
        Read Speed & $>$ 50 times/second \\
        \bottomrule
    \end{tabular}
\end{table}

% --- เริ่มส่วนแทรกรูปภาพ ---
\begin{figure}[h!]
    \centering
    % ปรับขนาดรูปตรง width (เช่น 0.6 คือ 60% ของความกว้างหน้ากระดาษ)
    \includegraphics[width=0.4\textwidth]{pictures/fm-505.png}
    \caption{RFID Module FM-505}
    \label{fig:FM505}
\end{figure}
% --- จบส่วนแทรกรูปภาพ ---

\begin{table}[h!]
    \centering
    \caption{คุณสมบัติของ RFID ID Tag  \cite{b_rfid_tag}}
    \label{tab:rfid-tag}
    \vspace{0.2cm}
    % เปลี่ยนเป็น tabularx และกำหนดความกว้างเท่ากับ \textwidth
    % l = คอลัมน์แรกชิดซ้ายตามความกว้างข้อความ
    % X = คอลัมน์สองขยายเต็มพื้นที่ที่เหลือและตัดบรรทัดอัตโนมัติ [2]
    \begin{tabularx}{0.5\textwidth}{lX} 
        \toprule
        \textbf{คุณสมบัติ (Property)} & \textbf{รายละเอียด (Value)} \\
        \midrule
        Brand Name & zhizaibide \\
        Model Number & LH-9662PST-100 \\
        Size & 73 x 21mm \\
        Frequency & UHF 860-960 MHz \\
        Read range & 1-15 m (depending on application/ reader/antenna/install sites) \\
        Protocol & ISO/IEC 18000-6c, EPC Class1 Gen2 \\
        Type & passive \\
        chip & alien 9662 U8 chip \\
        Function & Read and write \\
        EPC Storage & 96 bits \\
        User memory & 512 bits \\
        The service life & 100000 rewrite Times, 10 years of data storage capacity \\
        Using the environment & Temperature: - 50 $\sim$ + 85 $^\circ$C \\ % แก้ไขสัญลักษณ์ให้อ่านง่ายขึ้น
        Anti-jamming & strong \\
        Wide application & vehicle car access, books library, door control access, asset inventory, sports club clothing, warehouse, vehicle car, shop store, etc. \\
        Installation & sticker adhesive \\
        \bottomrule
    \end{tabularx}
\end{table}


    \item \textbf{Raspberry Pi 5:} ใช้เป็นหน่วยประมวลผลหลักในการควบคุมอุปกรณ์อ่าน RFID โดยทำการสื่อสารระหว่าง RFID Module (FM-505) และ Raspberry Pi 5 ผ่านทาง Protocol UART (Universal Asynchronous Receiver/Transmitter) และ จัดเก็บข้อมูลไว้ที่ระบบคลาวด์ โดยการสื่อสารเชื่อมโยงข้อมูลผ่านทาง API ที่พัฒนาด้วยภาษา Python  \cite{b_RaspberryPi_book}
    \item \textbf{Cloud Technology :}  สำหรับติดตั้ง Server และ จัดเก็บฐานข้อมูล โดยมี API ทำหน้าเชื่อมต่อระหว่างอุปกรณ์และ ระบบคลาวด์ \cite{b_aws}
    \item \textbf{Django Framework:} ใช้พัฒนา Web API ด้วยภาษา Python ประกอบด้วยการอ่านและบันทึกข้อมูลลงฐานข้อมูลบนคลาวด์ เพื่อมาใช้งานในชุดอ่าน RFID หรือ เครื่อง Client เช่น มือถือ และ คอมพิวเตอร์  \cite{b_Django_book}
    \item \textbf{Kivy : Cross-platform Python Framework for apps Development:} ใช้ในการออกแบบหน้าจอ GUI ที่ใช้ควบคุมอุปกรณ์และบริหารสินค้าคงคลังทั้งหมด \cite{b_kivy}
\end{enumerate}

% --- 3. การออกแบบระบบ ---
\section{การออกแบบและวิธีการดำเนินการ}
ระบบถูกออกแบบให้เชื่อมโยงระหว่างอุปกรณ์ฮาร์ดแวร์และระบบคลาวด์ ผ่านทาง API ที่พัฒนาขึ้นจาก Django ดังนี้:

\begin{figure}[h!]
    \centering
    % เรียกไฟล์รูปที่สร้างไว้ตะกี้เข้ามา
    %\documentclass[tikz]{standalone} % สั่งให้เป็นไฟล์แยกที่ compile ได้เอง

\usepackage{fontspec} 

% ตั้งค่าฟอนต์ภาษาไทย (เลือกฟอนต์ที่มีในเครื่องของคุณ)
% สำหรับ Windows แนะนำ: Leelawadee UI, Tahoma, หรือ Angsana New
% สำหรับ Mac แนะนำ: Thonburi
% สำหรับ Linux/Overleaf แนะนำ: Noto Sans Thai หรือ Sarabun
%\setmainfont{Thonburi} % <-- เปลี่ยนชื่อฟอนต์ตรงนี้ให้ตรงกับในเครื่องครับ


% *** เปลี่ยนชื่อฟอนต์ตรงนี้ถ้าเครื่องคุณใช้ฟอนต์อื่น ***
% เช่น Angsana New, Leelawadee UI, หรือ Browallia New
\setmainfont{TH Sarabun New}[
    Scale=1.25,       % ปรับขนาดฟอนต์ให้อ่านง่ายขึ้นใน format IEEE
    BoldFont={* Bold},
    ItalicFont={* Italic},
    BoldItalicFont={* Bold Italic}
]


\usepackage{fontawesome5}
\usetikzlibrary{shapes, arrows.meta, positioning, fit, backgrounds, shadows}

\begin{document}

\begin{tikzpicture}[
    node distance=2.5cm,
    every node/.style={font=\sffamily},
    device/.style={
        rectangle, 
        draw=gray!60, 
        thick, 
        rounded corners=5pt, 
        fill=white, 
        align=center, 
        minimum width=3cm, 
        minimum height=2cm,
        drop shadow
    },
    cloud_node/.style={
        cloud, 
        cloud puffs=15, 
        cloud puff arc=120, 
        draw=blue!50, 
        thick, 
        fill=blue!5, 
        aspect=2.5, 
        minimum width=6cm, 
        minimum height=4cm,
        align=center
    },
    api_box/.style={
        rectangle,
        draw=green!60!black,
        thick,
        fill=green!10,
        rounded corners,
        minimum width=2.5cm, 
        minimum height=1.5cm,
        align=center
    },
    link/.style={
        draw, 
        -Latex, 
        thick, 
        color=gray!80
    },
    label_text/.style={
        font=\footnotesize, 
        color=gray!90!black,
        midway,
        above
    }
]

    % --- 1. AWS Cloud Section ---
    \node[cloud_node] (aws) at (0,0) {}; 
    %\node[below=cm of aws.north] (aws_logo) {\Huge \faAws};
    %\node[below=0.1cm of aws_logo, font=\bfseries] {AWS Cloud};
    \node[below=1.2cm of aws.north, font=\bfseries] (aws_text) {Cloud};
    %\node[api_box, below=0.7 cm of aws_logo] (django) {  
    \node[api_box, below=0.2 cm of aws_text] (django) {      
        \Large \faPython \\ 
        \textbf{Django API} \\ 
        \footnotesize (RESTful)
    };

    % --- 2. Hardware Section ---
    \node[device, left=3cm of aws] (hardware) {
        \Huge \faMicrochip \\ 
        \vspace{0.2cm}
        \textbf{Hardware Unit} \\
        \footnotesize RPi 5 + RFID (FM-505)
    };

    % --- 3. Client Section ---
    \node[device, right=3cm of aws, yshift=1.5cm] (pc) {
        \Huge \faDesktop \\ 
        \vspace{0.2cm}
        \textbf{Web Dashboard} \\
        \footnotesize บริหารจัดการ
    };
    \node[device, right=3cm of aws, yshift=-1.5cm] (mobile) {
        \Huge \faMobile* \\ 
        \vspace{0.2cm}
        \textbf{Mobile App} \\
        \footnotesize ตรวจสอบสินค้าคงคลัง
    };

    % --- 4. Arrows ---
    \draw[link] (hardware) -- node[label_text] {HTTPS / JSON} (hardware -| aws.west);
    \draw[link, <->] (aws.east |- pc) -- node[label_text] {Response} (pc);
    \draw[link, <->] (aws.east |- mobile) -- node[label_text] {Response} (mobile);

\end{tikzpicture}

\end{document} 
    \resizebox{0.5\textwidth}{!}{%
        \input{graphs/rfinv_diagram_thai.tex}
    }%    
    \caption{สถาปัตยกรรมของระบบ}
    \label{fig:system_architecture}
\end{figure}


\subsection{ส่วนประกอบฮาร์ดแวร์}
ชุดอ่าน RFID ประกอบด้วย Raspberry Pi 5 เชื่อมต่อกับ RFID Reader Module (FM-505) เพื่อทำการอ่านค่าจาก RFID Tag ที่ติดอยู่กับสินค้าคงคลัง รหัสที่ได้จะถูกประมวลผลเบื้องต้นและทำการเชื่อมต่อไปยังระบบคลาวด์เพื่อทำการเก็บหรือเรียกค้นจากนั้นจะนำมาแสดงผลที่อุปกรณ์ผ่านทาง API ที่จัดทำขึ้น

% --- เริ่มส่วนแทรกรูปภาพ ---
\begin{figure}[h!]
    \centering
    % รูปซ้าย
    \begin{subfigure}[b]{0.2\textwidth}
        \centering
        \includegraphics[width=\textwidth]{scan_module_01_r.png}
        \caption{ชุดอ่าน RFID ด้านขวา} % แก้คำบรรยายตรงนี้
        \label{fig:scan01}
    \end{subfigure}
    \hfill % เว้นระยะห่าง
    % รูปขวา
    \begin{subfigure}[b]{0.22\textwidth}
        \centering
        \includegraphics[width=\textwidth]{scan_module_02_r.png}
        \caption{ชุดอ่าน RFID ด้านซ้าย} % แก้คำบรรยายตรงนี้
        \label{fig:scan02}
    \end{subfigure}
    
    \caption{แสดงลักษณะการติดตั้งชุดอ่าน RFID} % คำบรรยายรวม
    \label{fig:scan_modules}
\end{figure}
% --- จบส่วนแทรกรูปภาพ ---

\subsection{ส่วนประกอบซอฟต์แวร์}
พัฒนาระบบด้วยภาษา Python และ Django Framework ติดตั้งบน AWS EC2 โดยมีการจัดการฐานข้อมูลสินค้า (ชื่อ, ขนาด, สถานที่จัดเก็บ) API ผู้ใช้งานสามารถตรวจสอบสถานะสินค้า เพิ่มข้อมูล หรือดูรายงานผ่าน Web Browser บนคอมพิวเตอร์หรือโทรศัพท์มือถือได้

% --- เริ่มส่วนแทรกรูปภาพ ---
\begin{figure}[h!]
    \centering
    % ปรับขนาดรูปตรง width (เช่น 0.6 คือ 60% ของความกว้างหน้ากระดาษ)
    \includegraphics[width=0.5\textwidth]{pictures/django_db.png}
    \caption{ระบบบริหารฐานข้อมูลของ Django}
    \label{fig:my_single_image}
\end{figure}
% --- จบส่วนแทรกรูปภาพ ---

\subsection{ขั้นตอนการทำงาน}
1. ทำการลงทะเบียน RFID Tag เข้ากับสินค้าคงคลังด้วยชุดอ่าน RFID เพื่อเก็บข้อมูลไว้บนระบบคลาวด์ โดยจะทำการติด Barcode และ RFID Tag ไว้ที่ตัวสินค้าคงคลัง และจำลองสินค้าหลายขนาดต่างกันออกไป

% --- เริ่มส่วนแทรกรูปภาพ ---
\begin{figure}[h!]
    \centering
    % รูปซ้าย
    \begin{subfigure}[b]{0.2\textwidth}
        \centering
        \includegraphics[width=0.8\textwidth]{sample_01.png}
        \caption{ตัวอย่างที่ 1} % แก้คำบรรยายตรงนี้
        \label{fig:inv01}
    \end{subfigure}
    \hfill % เว้นระยะห่าง
    % รูปขวา
    \begin{subfigure}[b]{0.23\textwidth}
        \centering
        \includegraphics[width=1\textwidth]{sample_02.png}
        \caption{ตัวอย่างที่ 2} % แก้คำบรรยายตรงนี้
        \label{fig:inv02}
    \end{subfigure}
    
    \caption{แสดงตัวอย่างของการติดรหัส และ ขนาดของสินค้าคงคลัง} % คำบรรยายรวม
    \label{fig:invs}
\end{figure}
% --- จบส่วนแทรกรูปภาพ ---


2. ใช้ชุดอ่าน RFID ทำการอ่านรหัสสินค้าคงคลังเพื่อตรวจสอบความถูกต้องเทียบกันระหว่างระบบเดิมซึ่งใช้ Barcode และ ระบบใหม่ที่ใช้ RFID Tag ในการระบุรหัสสินค้าคงคลัง

% --- เริ่มส่วนแทรกรูปภาพ ---
\begin{figure}[h!]
    \centering
    % ปรับขนาดรูปตรง width (เช่น 0.6 คือ 60% ของความกว้างหน้ากระดาษ)
    \includegraphics[width=0.5\textwidth]{pictures/tag_test.png}
    \caption{การทดสอบการอ่านรหัสสินค้าคงคลัง}
    \label{fig:tag_test}
\end{figure}
% --- จบส่วนแทรกรูปภาพ ---

3. ข้อมูลการตรวจสอบจะถูกบันทึกขึ้นสู่ระบบที่ออกแบบไว้บนระบบคลาวด์ และนำมาใช้ในการคำนวณ 

\subsection{สูตรที่ใช้ในการคำนวณ} 
ในการทดสอบประสิทธิภาพการอ่านข้อมูล ได้กำหนดตัวแปรและสูตรการคำนวณดังนี้:

% สูตรหาค่าเฉลี่ย
\noindent \textbf{1.1 สูตรหาค่าเฉลี่ยเวลาในการอ่าน (Average Time)}
\begin{equation}
    \bar{t} = \frac{\sum_{i=1}^{n} t_i}{n}
\end{equation}
โดยกำหนดให้:
\begin{itemize}
    \item $\bar{t}$ คือ เวลาเฉลี่ยในการหารหัสสินค้า มีหน่วยเป็นวินาที
    \item $t_i$ คือ เวลาที่ใช้ในการทดสอบรอบที่ $i$
    \item $n$ คือ จำนวนรอบการทดสอบ (ในที่นี้ $n=5$)
\end{itemize}

\vspace{0.2cm}

% สูตรหาความเร็ว
\noindent \textbf{1.2 ความเร็วในการอ่านข้อมูล (Reading Speed)}
\begin{equation}
    Speed = \frac{N}{\bar{t}}
\end{equation}
โดยกำหนดให้:
\begin{itemize}
    \item $Speed$ คือ ความเร็วในการอ่านรหัสสินค้างคงคลัง (รหัส/วินาที)
    \item $N$ คือ จำนวนรหัสสินค้าคงคลังที่นำมาทดสอบ (5, 10, 30, 50, 100 ชิ้น)
    \item $\bar{t}$ คือ เวลาเฉลี่ยจากการคำนวณ
\end{itemize}



% --- เริ่มส่วนแทรกรูปภาพ ---
\begin{figure}[h!]
    \centering
    % ปรับขนาดรูปตรง width (เช่น 0.6 คือ 60% ของความกว้างหน้ากระดาษ)
    \includegraphics[width=0.5\textwidth]{pictures/rf_time.png}
    \caption{ผลการทดสอบความเร็วในการอ่านรหัสสินค้าคงคลัง}
    \label{fig:time_test}
\end{figure}
% --- จบส่วนแทรกรูปภาพ ---

%\vspace{0.2cm}

% --- 4. ผลการทดลอง (ส่วนที่เพิ่มตามโจทย์) ---
\section{ผลการทดลองและการวิเคราะห์}
ในการศึกษาครั้งนี้ ได้ทำการทดลองเปรียบเทียบประสิทธิภาพระหว่างการใช้ระบบ Barcode แบบดั้งเดิมในที่นี้จะใช้เครื่องอ่าน Barcode แบบเลเซอร์ Zebra LS2208 \cite{b_ZebraLS2208} เปรียบเที่ยบกับชุดอ่าน RFID ที่พัฒนาขึ้น โดยวัดเวลาที่ใช้ในการตรวจนับสินค้าคงคลังในจำนวนที่ต่างกัน (5 \ref{tab:scan_time5}, 10, 30, 50, และ 100 ชิ้น)

\subsection{ผลการเปรียบเทียบด้านเวลาที่ใช้ในการอ่านรหัสสินค้าคงคลัง}
ผลการทดลองแสดงความสัมพันธ์ระหว่างจำนวนสินค้าคงคลังและเวลาที่ใช้ในการตรวจสอบ ดังนี้


% --- ส่วนเนื้อหา (เรียกใช้งานครบทั้ง 5 ตาราง) ---

\RFIDHalfTable{5}{0.81}{0.75}{0.58}{0.63}{0.92}{0.74}{9.34 & 7.24 & 8.62 & 8.68 & 8.34 & \textbf{8.37}}
\label{tab:scan_time5}

\RFIDHalfTable{10}{0.86}{1.21}{0.73}{0.92}{1.53}{1.05}{14.26 & 13.37 & 12.68 & 13.49 & 14.35 & \textbf{13.63}}

\RFIDHalfTable{30}{5.24}{4.69}{6.62}{5.42}{5.65}{5.52}{43.28 & 39.42 & 42.36 & 37.49 & 38.24 & \textbf{40.16}}

\RFIDHalfTable{50}{7.43}{8.94}{9.71}{8.65}{8.21}{8.59}{72.43 & 82.38 & 76.46 & 74.85 & 78.69 & \textbf{76.96}}

\RFIDHalfTable{100}{26.42}{25.43}{22.46}{23.34}{22.94}{24.12}{142.35 & 139.24 & 146.49 & 143.49 & 141.92 & \textbf{142.70}}


\begin{figure}[htbp]
\centering

\scalebox{0.9}{ 
\begin{tikzpicture}
\begin{axis}[
    title={กราฟเปรียบเทียบเวลาที่ใช้ในการตรวจนับสินค้าคงคลัง},
    xlabel={จำนวนสินค้าคงคลัง (ชิ้น)},
    ylabel={เวลาที่ใช้ (วินาที)},
    xmin=0, xmax=100,
    ymin=0, ymax=160,
    xtick={0,5,10,30,50,100},
    ytick={0,20,40,60,80,100,120,140,160},
    legend pos=north west,
    ymajorgrids=true,
    grid style=dashed,
    legend style={font=\footnotesize} % ปรับขนาดฟอนต์ใน legend
]

% กราฟ RFID (เส้นตรง ความชันน้อย)
\addplot[
    color=blue,
    mark=square,
    thick
    ]
    coordinates {
    (5, 0.74)
    (10, 1.05)
    (30, 5.52)
    (50, 8.59)
    (100, 24.12)
    };
    \addlegendentry{ระบบ RFID}

% กราฟ Barcode (เส้นตรง ความชันสูง)
\addplot[
    color=red,
    mark=triangle,
    thick
    ]
    coordinates {
    (5, 8.37)
    (10, 13.63)
    (30, 40.16)
    (50, 76.96)
    (100, 142.70)
    };
    \addlegendentry{ระบบ ฺBarcode}

\end{axis}
\end{tikzpicture}
}
\caption{เปรียบเทียบความเร็วในการตรวจนับสินค้า ระบบ RFID (เส้นสีน้ำเงิน) ใช้เวลาเพิ่มขึ้นเพียงเล็กน้อยเมื่อสินค้าเพิ่มขึ้น ในขณะที่ระบบ Barcode (เส้นสีแดง) ใช้เวลาเพิ่มขึ้นแบบทวีคูณ}
\label{fig:comparison}
\end{figure}


\subsection{ผลการทดสอบการอ่านรหัสสินค้าคงคลัง}
เนื่องจากหลักการทำงานของ RFID คือใช้คลื่นวิทยุไม่ได้ใช้แสงในการอ่านรหัสเหมือน Barcode และ QR-Code จึงทำการทดลองประสิทธิภาพในการอ่านรหัสสินค้าคงคลังผ่านบรรจุภัณฑ์ต่างชนิดกัน ในการทดลองครั้งนี้จะนำกล่องกระดาษ และกล่องพลาสติกมาบรรจุสินค้า และทำการเปรียบเทียบความเร็วในการอ่านรหัสว่าแตกต่างกันหรือไม่ ซึ่งผลการทดลองแสดงได้ในตารางนี้ 

\BoxTable{5}{0.92}{0.71}{0.76}{0.85}{0.79}{0.81}{0.84 & 0.62 & 0.78 & 0.87 & 0.68 & \textbf{0.76}}

% --- เริ่มส่วนแทรกรูปภาพ ---
\begin{figure}[h!]
    \centering
    % รูปซ้าย
    \begin{subfigure}[b]{0.4\textwidth}
        \centering
        \includegraphics[width=\textwidth]{box1.png}
        \caption{การจัดเรียงสินค้าในกล่องกระดาษ} % แก้คำบรรยายตรงนี้
        \label{fig:box1}
    \end{subfigure}
    \hfill % เว้นระยะห่าง
    % รูปขวา
    \begin{subfigure}[b]{0.35\textwidth}
        \centering
        \includegraphics[width=\textwidth]{box2.png}
        \caption{ปิดกล่องเพื่อทำการทดสอบ} % แก้คำบรรยายตรงนี้
        \label{fig:box2}
    \end{subfigure}
       % รูปซ้าย
    \begin{subfigure}[b]{0.4\textwidth}
        \centering
        \includegraphics[width=\textwidth]{plastic_box1.png}
        \caption{การจัดเรียงสินค้าในกล่องพลาสติก} % แก้คำบรรยายตรงนี้
        \label{fig:box3}
    \end{subfigure}
    \hfill % เว้นระยะห่าง
    % รูปขวา
    \begin{subfigure}[b]{0.35\textwidth}
        \centering
        \includegraphics[width=\textwidth]{plastic_box2.png}
        \caption{ปิดกล่องเพื่อทำการทดสอบ} % แก้คำบรรยายตรงนี้
        \label{fig:box4}
    \end{subfigure} 
    \caption{แสดงการเรียงสินค้าในกล่องกระดาษและกล่องพลาสติกเพื่อทำการทดสอบประสิทธิภาพการอ่านรหัสสินค้าคงคลังในบรรจุภัณฑ์ต่างชนิดกัน} % คำบรรยายรวม
    \label{fig:boxs}
\end{figure}
% --- จบส่วนแทรกรูปภาพ ---


\subsection{ความคุ้มค่าในการนำ RFID มาแทน Barcode หรือ QR-Code}
ถึงแม้ว่าการนำระบบ RFID มาประยุกต์ใช้งานจะมีต้นทุนด้านอุปกรณ์เริ่มต้นที่สูงกว่าระบบ Barcode และ QR-Code แต่เมื่อพิจารณาเวลาในการใช้งานที่ลดลงและความแม่นยำในการอ่านรหัสจำนวนมาก ระบบ RFID จึงให้ความคุ้มค่าที่สูงกว่ามาก ดังที่แสดงในตาราง \ref{fig:compare_tag}

\begin{table}[h]
\centering
\renewcommand{\arraystretch}{1.5} % เพิ่มความสูงของแถวให้อ่านง่ายขึ้น
\begin{tabular}{|p{2.2cm}|p{1.7cm}|p{1.7cm}|}
\hline
% ใส่สีแถวหัวตาราง (เทาอ่อน) [3]
\rowcolor{lightgray} 
\textbf{หัวข้อเปรียบเทียบ} & \textbf{ระบบ Barcode หรือ QR-Code} & \textbf{ระบบ RFID}  \\ \hline
ต้นทุนต่อหน่วย  & ต่ำมาก & สูงกว่า  \\ \hline
ต้นทุนอุปกรณ์อ่าน & ราคาถูก หาซื้อง่าย & ราคาสูง \\ \hline
อายุการใช้งาน & สั้น ชำรุดได้ง่าย & ยาวนานกว่า  \\ \hline
ต้นทุนด้านแรงงาน & สูง ต้องทำการตรวจทีละชิ้น & ต่ำ อ่านได้ครั้งละมากๆได้    \\ \hline
ความแม่นยำของข้อมูล & มีโอกาสผิดพลาดจากผู้ใช้ & สูงกว่า  \\ \hline
\end{tabular}
\caption{ตารางเปรียบเทียบระหว่าง Barcode หรือ QR-Code และ ระบบ RFID }
\label{fig:compare_tag}
\end{table}



% --- 5. สรุปผล ---
\section{สรุปผลการศึกษา}
การพัฒนาระบบบริหารจัดการสินค้าคงคลังด้วย RFID และเทคโนโลยีคลาวด์ ช่วยแก้ปัญหาความล่าช้าและความยุ่งยากของระบบ Barcode และ QR-Code ในระบบดั้งเดิมได้อย่างมีประสิทธิภาพ จากผลการทดลองยืนยันว่า RFID ช่วยลดเวลาในการตรวจสอบสินค้าได้อย่างมีนัยสำคัญ โดยเฉพาะในการบริหารจัดการสินค้าจำนวนมาก อีกทั้งการใช้ระบบคลาวด์ยังช่วยให้ข้อมูลมีความปลอดภัยและเข้าถึงได้ง่าย ลดภาระค่าใช้จ่ายในการดูแล Server และการใช้พลังงานไฟฟ้าภายในองค์กรได้อีกด้วย แต่ยังมีข้อจำกัดทางกายภาพที่ต้องคำนึงถึงในการใช้ RFID เนื่องจากลักษณะการทำงานที่เป็นคลื่นวิทยุเมื่ออยู่ใกล้วัสดุที่เป็นโลหะหรือมีของเหลว จะทำให้ประสิทธิภาพในการสื่อสารระหว่าชุดอ่านและตัว RFID Tag ลดลง และการจัดเรียงสินค้า ต้องไม่ชิดกันแน่นจนเกินไป \cite{my_paper}


\printbibliography % แสดงรายการอ้างอิง 


\end{document}